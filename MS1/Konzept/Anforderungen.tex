\chapter{Anforderungen}
\begin{itemize}
\section{Funktionale Anforderungen}

	\item Das System muss Berechnungen f�r den Aquarium Halter durchf�hren, welche den durchschnittlichen N�hrstoffverbrauch, die darauf basierende ben�tigte Menge an D�ngemittel, die Verdunstungsmenge des Wassers und den CO2 Gehalt berechnen kann.
	\item Das System muss dem Halter des Aquariums erm�glichen, gezielte Wasserwechsel durchzuf�hren.
	\item Das System sollte die Verwaltung von mehreren Aquarien erm�glichen.
	\item Das System muss eine sichere Verwaltung der Nutzerdaten garantieren.
	\item Das System muss dem Halter des Aquariums die M�glichkeit geben, selbst die Wasserwerte einzutragen.
	\item Das System muss �ber eine eindeutige Verbindung zwischen Halter des Aquariums und dem Fachhandel verf�gen.
	\item Das System sollte die Ver�nderung der N�hrwerte im Wasser �bersichtlich veranschaulichen.
	\item Das System sollte die Informationen �ber das Aquarium sowohl f�r Halter der Aquarien und den Fachhandel stets aktuell halten.
	\item Das System soll dem Halter des Aquariums eine D�ngeempfehlung basierend auf den Wasserwerten geben.
	\item Das System sollte die aktuellen N�hrstoffwerte mit Hilfe des durchschnittlichen N�hrstoffverbrauchs sch�tzen k�nnen.
	\item Bei der Erstanmeldung soll der Aquarienbesitzer F�llmenge und Abmessungen des Aquariums im System speichern k�nnen
	\item Das System soll dem Aquarium Besitzer erm�glichen, bei Problemen direkte Hilfe vom Fachmarkt zu erhalten.
	\item Sobald ein Aquarium Halter ein neues Objekt f�rs Aquarium gekauft hat, soll er dieses in sein virtuelles Aquarium hinzuf�gen.
	\item Das System soll dem Fachh�ndler die individuellen Kundendaten anzeigen k�nnen.
	\item Das System wird dem Halter die M�glichkeit geben, die Aquarium Bestandteile in ein virtuelles Aquarium einzutragen.
\section{Nicht-Funktionale Anforderungen}

	\item Das System sollte regelm��ig die Daten als Backup speichern.
	\item Das System sollte zeitunabh�ngig genutzt werden k�nnen.
	\item Das System sollte dem Aquarium Besitzer erm�glichen, nur bestimmte Daten weiter zu geben.
	\item Das System soll bestm�gliche Gebrauchstauglichkeit erm�glichen.
	\item Das System soll korrekte Ergebnisse liefern.
	\item Das System soll eine m�glichst nahe N�hrwertsch�tzung liefern.
	\item Das System soll dem Nutzer eine bessere Betreuung durch den Fachhandel geben.
\end{itemize}

Definitionen: 

Berechnungen:
Halter des Aquariums: 
Kundendaten:
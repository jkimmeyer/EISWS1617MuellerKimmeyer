\chapter{Kommunikationsmodell}
Bei den Kommunikationsmodellen l�sst sich zwischen dem deskriptiven und dem pr�skriptiven Modell unterscheiden. Diese zeigen den Ist-Zustand und den Soll-Zustand der Kommunikationswege. Beim deskriptiven Modell sieht man, dass Kunde und Fachh�ndler auf direktem Wege kommunizieren. Zum Erhalt der Ergebnisse der Wasserprobe ist also vermutlich ein weiterer Gang zur Fachhandlung n�tig. Im pr�skriptiven Modell sieht man, dass dieser Weg erspart bleibt, da die Kommunikation nun �ber das System gehandhabt wird. Au�erdem hat der Fachh�ndler direkt �ber das System die M�glichkeit Berechnungen auszuf�hren. Der Kunde hat ebenfalls nun diese M�glichkeiten und wird durch das System sogar zur Benutzung dieser Berechnungen angeregt. Dann sieht man noch in beiden Modellen den Wissenschaftler, der zu den genauen Berechnungen beitr�gt. F�r ihn �ndert sich durch das System nichts.
\section{Deskriptives Kommunikationsdiagramm}

\begin{figure}[ht!]
\centering
\includegraphics[scale=0.45]{../workinprogress/Kommunikationsdiagramm_deskriptiv.jpg}
\captionbelow[Deskriptives Kommunikationsdiagramm]{Deskriptives Kommunikationsdiagramm}
\label{desk_kom}
\end{figure}

\section{Pr�skriptives Kommunikationsdiagramm}

\begin{figure}[ht!]
\centering
\includegraphics[scale=0.45]{../workinprogress/Kommunikationsdiagramm_praeskriptiv.jpg}
\captionbelow[Pr�skriptives Kommunikationsdiagramm]{Pr�skriptives Kommunikationsdiagramm}
\label{pr�sk_kom}
\end{figure}
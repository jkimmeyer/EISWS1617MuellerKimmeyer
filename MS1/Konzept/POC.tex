\chapter{Proof of Concept}
\section{Risiken}
\textbf{Beschaffung der Formeln f�r einzelne Berechnungen}\\
Die Berechnungen, die wir in unserem System benutzen wollen, sind nicht gerade einfach. Diese m�ssen erst recherchiert werden. Sollten wir nicht an die Formeln f�r einzelne Berechnungen gelangen, fehlt ein wichtiger Teil unserer Anwendungslogik. Dies kann dazu f�hren, dass unsere Anwendungslogik nicht mehr ausreicht und wir im schlimmsten Fall das Projekt beenden m�ssen. Um das Risiko zu minimieren sollten wir uns schon fr�hzeitig um die Formeln k�mmern und uns Alternativen �berlegen.\\[0,5cm] 
\textbf{Umsetzung der Formeln (Programmierung)}\\
Sollten wir alle Formeln bekommen haben gilt es diese in Java umzusetzen. Dies k�nnte aufgrund von mangelnden Programmierkenntnissen oder auch durch Faktoren wie zum Beispiel die Unwissenheit �ber den nat�rlichen N�hrstoffverbrauch im Aquarium scheitern und das k�nnte wiederum zu einem Abbruch des Projekts f�hren, da uns in dem Fall einiges an Anwendungslogik verloren geht. Um dies fr�hzeitig zu erkennen ist hier ein Proof of Concept sinnvoll.\\[0,5cm]
\textbf{Umgang mit Firebase Cloud Messaging}\\
Da bis jetzt noch niemand von uns mit dem FCM gearbeitet hat, ist es nicht garantiert, dass wir unser Vorhaben damit umsetzen k�nnen. Sollten wir nicht in der Lage sein, die Kommunikation mit dem FCM umsetzen zu k�nnen, m�ssen wir darauf verzichten und eine alternative M�glichkeit f�r die Kommunikation zwischen den Clients suchen. Zur fr�hzeitigen Erkennung des Ereignisses ist auch hier ein Proof of Concept angebracht.\\[0,5cm]
\textbf{Ausfall eines Teammitglieds (z.B. durch Krankheit)}\\
Sollte einer von uns beiden zum Beispiel durch Krankheit f�r mehrere Tage ausfallen, steht das Projekt auf der Kippe, da die Bearbeitung der Artefakte alleine um einiges l�nger dauert und wir somit unseren Zeitplan nicht mehr einhalten k�nnen.\\[0,5cm]
\textbf{Untersch�tzung des Zeitaufwands}\\
Das Einhalten der Fristen ist ein wichtiger Punkt, da das Projekt ansonsten scheitern k�nnte. Es ist also n�tig, den Zeitaufwand von Anfang an richtig einzusch�tzen und falls m�glich noch ein bisschen Freiraum bis zum Abgabetermin einzuplanen, um Verz�gerungen ausgleichen zu k�nnen.\\[0,5cm]
\textbf{Fehlendes MCI Wissen}\\
Da wir beide noch keine MCI Pr�fung absolviert haben, kann es sein, dass uns das Wissen bei der Bearbeitung der MCI-Artefakte fehlt. Um dies so gut wie m�glich zu verhindern haben wir bereits am Anfang des Semesters begonnen, die MCI Materialien zu wiederholen.\\ 
\section{Alternativen}
\section{Proof of Concept}
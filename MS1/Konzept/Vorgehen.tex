\chapter{Methodischer Rahmen - MCI}
\section{Abw�gung der Vorgehensmodelle}
Bei der Wahl des Vorgehensmodell wird entschieden, worauf wir den Fokus legen. Entweder auf den Benutzer beim user-centered-design, wenn man die Merkmale des Benutzers als sinnvolle Grundlage f�r das System erachtet oder auf die Funktionalit�t dem Usage-centered-design, wenn die Arbeit und das Erreichen der Ziele der Arbeit im Vordergrund steht und die Benutzung als Ausgangspunkt f�r die Modellierung gew�hlt werden kann.

Da wir bereits von Anfang an beschlossen haben, dass wir ein besonderes Augenmerk darauf haben wollen, die Aquaristik f�r Menschen, welche eher weniger Erfahrung und Wissen im Bereich der Aquaristik besitzen, interessanter zu machen, viel unsere Wahl auf einen User-centered-design Prozess. Da das System eine Kommunikation zwischen Kunden und Fachhandel besitzt, wird der Fokus auf dem Benutzer noch bedeutender. 
\subsection{Usability Engineering Lifecycle}
Der Usability Engineering Lifecycle von Deborah Mayhew ist ein skalierbares Vorgehensmodell, welches ein individuelles Eingehen auf das vorliegende Projekt erm�glicht. Der Prozess ist strukturiert und iteriert die jeweils abgeschlossenen Level des Prozesses. Der Nachteil an diesem Modell findet sich dadrin, dass es einer umfangreichen Anforderungsanalyse nicht ganz gerecht wird und die R�ckmeldung der einzelnen Benutzer erst im dritten Schritt erfolgt.
\subsection{Szenariobasiertes Vorgehensmodell}
Wir haben uns gegen dieses Modell von Rosson und Carrol entschieden, da das Modell nicht skalierbar ist. Die Erstellung der einzelnen Szenarios ist sehr aufwendig und aufgrund der Gr��e unseres Projekts f�r uns nicht von Bedeutung ist. Die Entwicklung von Alternativen ist bei diesem Vorgehensmodell auch eingeschr�nkt, da immer ein Zusammenhang zwischen den technischen Subsystemen und der Interaktion vorhanden ist.
\subsection{Discount-Usability Engineering}
Nielsen publizierte ein vereinfachtes Vorgehensmodell, welches sich auf der Annahme st�tzt, dass schon mit seinen drei vereinfachten Schritten eine erhebliche Verbesserung der Gebrauchstauglichkeit zu erreichen ist. Wir haben uns gegen dieses Modell entschieden, da wir vermutlich Probleme damit h�tten, eine angemessene Anzahl an Probanden zu finden, auch wenn diese schon sehr reduziert ist. Zudem erscheint uns dieses Modell f�r die Konzeption unseres Projektes ein bisschen zu oberfl�chlich.
\subsection{Prozess zur Gestaltung gebrauchstauglicher interaktiver Systeme - DIN EN ISO 9241, Teil 210}
Da es sich hier um eine Norm handelt, ist dieses Vorgehensmodell zun�chst sehr allgemein. Durch eine hohe Skalierbarkeit l�sst es sich aber individuell an ein Projekt anpassen. Des weiteren profitiert man von einer sehr ausf�hrlichen und verst�ndlich erkl�rten Dokumentation des Vorgehens. Diese Methode ist sehr umfangreich, da zum Beispiel auch bei jeder Anforderungs�nderung die menschzentrierten Gestaltungsaspekte wieder �berarbeitet werden m�ssen. Einen weiteren Vorteil sehen wir in der �berpr�fung der erfolgreichen Gestaltung des gebrauchstauglichen Systems anhand der vorhandenen Checkliste der DIN EN ISO 9241, Teil 210.

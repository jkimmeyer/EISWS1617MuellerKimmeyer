\chapter{Anforderungen}
Die Anforderungen bilden die Grundlage für unser System und gehen aus den Erfordernissen und Erwartungen der Stakeholder hervor. Für das Konzept haben wir eine sehr grobe Anforderungsübersicht erstellt, welche wir im weiteren Verlauf nochmal genauer spezifizieren werden.

Die erste Iteration der Anforderungen ging mit den gesetzten Usability Goals nach Deborah Mayhew einher.

\section{Funktionale Anforderungen}

	F10: Das System muss Berechnungen für den Aquarium Halter durchführen, welche den durchschnittlichen Nährstoffverbrauch von NO3, PO4, FE und Kalium, die darauf basierende benötigte Menge an Düngemittel, die Verdunstungsmenge des Wassers und den CO2 Gehalt berechnen kann.\\
	F20: Das System muss dem Halter des Aquariums ermöglichen, durch Berechnungen gezielte Wasserwechsel durchzuführen.\\
	F30: Das System muss dem Halter des Aquariums die Möglichkeit geben, selbst die Wasserwerte einzutragen.\\
	F40: Das System sollte die Veränderung der Nährwerte im Wasser übersichtlich veranschaulichen.\\
	F50: Das System sollte die Informationen über das Aquarium sowohl für Halter der Aquarien und den Fachhandel stets aktuell halten.\\
	F60: Das System soll dem Halter des Aquariums eine Dosiermenge des Düngemittels basierend auf den Wasserwerten geben.\\
	F70: Das System sollte die aktuellen Nährstoffwerte mit Hilfe des durchschnittlichen Nährstoffverbrauchs schätzen können.\\
	F80: Bei der Erstanmeldung soll der Aquarienbesitzer Füllmenge und Abmessungen des Aquariums im System speichern können.\\
	F90: Das System soll dem Aquarium Besitzer ermöglichen, bei Problemen direkte Hilfe vom Fachmarkt zu erhalten.\\
	F100: Sobald ein Aquarium Halter ein neues Objekt fürs Aquarium gekauft hat, soll es direkt über den Fachhandel in sein virtuelles Aquarium hinzugefügt werden.\\
	F110: Das System soll dem Fachhändler die individuellen Kundendaten, wie Name, Adresse, Kontaktdaten und Einkäufe anzeigen können.\\
	F120: Das System soll dem Halter die Möglichkeit geben, die Aquarium Bestandteile wie Fische, Wasserpflanzen, Lampen etc.  in ein virtuelles Aquarium einzutragen.\\
	F130: Das System muss dem Aquarianer die Möglichkeit geben, die Aquarium Bestandteile zu ändern.

\section{Qualitative Anforderungen}

	Q10: Das System sollte regelmäßig die Daten als Backup speichern.\\
	Q20: Das System sollte zeitunabhängig genutzt werden können.\\
	Q30: Das System soll bestmögliche Gebrauchstauglichkeit ermöglichen.\\
	Q40: Das System soll korrekte Ergebnisse liefern.\\
	Q50: Das System soll eine möglichst nahe Nährwertschätzung liefern.\\
	Q60: Das System soll dem Nutzer eine bessere Betreuung durch den Fachhandel geben.\\
	
	
\section{Organisationale Anforderungen}
	O10: Das System sollte dem Aquarium Besitzer ermöglichen, nur bestimmte Daten weiter zu geben.	\\
	O20: Das System soll dem Kunden ermöglichen, mehrere Aquarien zu verwalten.\\
	O30: Der Fachhandel soll an noch offene Wasseranalysen aufmerksam gemacht werden.\\
	O40: Das System muss eine sichere Verwaltung der Nutzerdaten garantieren.\\
	O50: Das System muss über eine eindeutige Verbindung zwischen Halter des Aquariums und dem Fachhandel verfügen.\\

Definition von Wasserwerte/ Nährwerte:

\begin{table}[]
\centering
\caption{My caption}
\label{my-label}
\resizebox{\textwidth}{!}{%
\begin{tabular}{|p{4cm}|p{4cm}|p{6cm}|p{3cm}|p{4cm}|l}
\hline
\textbf{Actor} & \textbf{Goal} & \textbf{Brief Description} & \textbf{Trigger} & \textbf{Problem} \\ \hline
\multirow{7}{*}{Aquarienbesitzer} & Dokumentation der Wasserwerte & Der Besitzer eines Aquariums bringt eine Wasserprobe zum Fachhändler, welche diese analysiert und die Ergebnisse dem Besitzer zur permanenten Speicherung zurückgibt. & Probleme im Aquarium/ optimale Bedingungen sind dem Aquarianer wichtig & Dokumentation ist umständlich. Eintragen ist aufwendig \\ \cline{2-5} 
 & Dokumentation der Wasserwerte & Der Besitzer besitzt ein Tool zur Wasseranalyse und dokumentiert anschließend die Werte permanent. & Probleme im Aquarium/ optimale Bedingungen sind dem Aquarianer wichtig & Eintragen der Wasserwerte aufwendig. Papier ist nicht sehr wasserbeständig. Spritzwassergefahr!! \\ \cline{2-5} 
 & Düngung des Wassers & Basierend auf den Idealwerten der Nährstoffe und der aktuellen Wasserwerte wird die optimale Düngerdosierung berechnet und der Besitzer fügt die angemessene Menge an Düngemittel dem Wasser hinzu. & Es fehlt ein bestimmter Nährstoff im Wasser & Berechnung der Dosierung und das Kennen der Idealwerte sehr umständlich. Fehlendes Wissen \\ \cline{2-5} 
 & Fachhändler das Aquarium präsentieren & Der Besitzer möchte die einzelnen Komponenten, wie Fische, Pflanzen, Lampen und Boden dem Fachhandel präsentieren, damit er eine bessere Beratung vornehmen kann. & Kauf neuer Objekte/ Beratung durch Fachhandel/ Probleme im Aquarium & Fotos und Beschreibungen oft ungenau, unvollständig oder auch zu umständlich. \\ \cline{2-5} 
 & Zielgerichteten Wasserwechsel durchführen & Anhand der Werte des Leitungswasser wird das Verhältnis zwischen Osmose- und Leitungswasser für die richtige Zielmenge berechnet. Der Besitzer muss anschließend diesen WW durchführen. & Wöchentliche Aquariumpflege / Falsche Wasserwerte & Berechnung ist aufwendig. \\ \cline{2-5} 
 & Probleme behandeln & Bei Problemen die nicht auf die Wasserwerte zurückzuführen sind, muss der Aquarianer einen Mitarbeiter aus dem Fachhandel zu sich nach Hause bestellen. So kann dieser vor Ort die Umstände prüfen. & Problemfindung auf Entfernung nicht möglich & Die Fahrt zu einem Kunden nach Hause ist sehr kostenintensiv und zeitaufwendig. Termine sind oft nur in weiter Entferung zu erhalten. \\ \cline{2-5} 
 & Kauf von Objekten fürs Aquarium & Der Kunde kauft im Fachhandel neue Objekte fürs Aquarium, wie zum Beispiel Fische oder Wasserpflanzen. Dabei muss dieser darauf achten, dass die gekauften Objekte zu den übrigen Umständen passen. & Dem Aquarianer fehlt etwas im Aquarium & Der Kauf von Objekten passt oft nicht zu dem individuellen Aquarium. \\ \hline
\multirow{4}{*}{Fachhandel} & Analyse der Wasserwerte & Der Kunde(Aquarienbesitzer) bringt die Probe zum Fachhandel, der Fachhandel ist anschließend dafür verantwortlich, die Proben zu analysieren und die Werte an den Kunden weiterzugeben. & Kunde möchte eine Wasseranalyse & Zeitliche Dauer, der Kunde muss oft zweimal zum Fachhandel \\ \cline{2-5} 
 & Beratung des Kunden aufgrund von Problemen & Fachhandel schaut sich die Wasserwerte der Kunden an und reagiert daraufhin mit Empfehlung des optimalen Düngemittels. Die Dosierung übernimmt der Kunde selbst. & Schlechte Wasserwerte im Aquarium des Kunden & Dokumentation der Wasserwerte und Empfehlungen auf Papier geht oft verloren. Verlangt Anwesenheit des Kundens im Geschäft \\ \cline{2-5} 
 & Optimale Beratung des Kunden & Empfehlungen von Fischen und Pflanzen anhand der analysierten Wasserwerte. Der Kunde erhält so eine individuelle Analyse und der Fachhändler kann entsprechend der Wasserwerte auf Probleme reagieren. & Kunde möchte beraten werden & Individuelle Aquarienübersicht der Kunden schwierig zu überblicken, da die Kunden oft wechseln und die Wasserwerte nur von einmaliger Messung vorliegen \\ \cline{2-5} 
 & Problemanalyse beim Kunden & Anhand der Wasserwerte oder auch Besuch bei einem Kunden kann das Aquarium auf alle möglichen Einflüsse, die zu Problemen führen, überprüft werden. & Problemfindung auf Entfernung nicht möglich & Die Fahrt zu einem Kunden nach Hause ist sehr kostenintensiv und zeitaufwendig. Termine sind oft nur in weiter Entferung zu erhalten. \\ \hline
\end{tabular}%
}
\end{table}
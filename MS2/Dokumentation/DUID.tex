\chapter{Detailed User Interface Design}
Das Ziel aller vorangegangenen Prozesse ist das Detailed User Interface Design. Laut Mayhew führt die Berücksichtigung aller relevanten Arbeitsschritte in diesem finalen Prozess zu einem Ergebnis, welches in den Bereichen der User Performance und Zufriedenheit weit vorne steht. \cite[325]{Mayhew:UEL}\\

Mayhew schlägt vor zunächst alle Navigationspfade zu realisieren, damit jedes Element aufgerufen werden kann. Anschließend werden alle Fenster, Dialogboxen und Nachrichtenboxen mit Hilfe der Standard Screen Designs vervollständigt. Zum Schluss wird noch auf alle möglichen Interaktion zwischen Eingabegeräten und dem System hingewiesen. \\

Entgegen der Empfehlung von Deborah Mayhew, beim Detailed User Interface Design alle Screens zu visualisieren, haben wir uns entschieden, redundante Screens, welche ähnliche Funktionen haben, aufgrund des Projektrahmens nicht extra zu modellieren.
Aufgrund des Umfangs ist der komplette UI-Prototyp im Anhang zu finden.\\

\section{Android - Anwendung}
Auch hier findet erneut die Unterteilung zwischen der mobilen Anwendung und der Desktopanwendung statt. Da sich die beiden Medien grundlegend unterscheiden, besteht hier auch gar keine andere Möglichkeit.
\subsection{Navigationspfade}

Zu Anfang hatten wir bei den Navigationspfaden noch die Wahl zwischen dem Modell \ref{kmodell21} mit einem Seitenmenü und dem Modell \ref{kmodell31}. Hier wurde relativ schnell klar, dass wir zur zweiten Alternative neigen, um das schon trockene Thema etwas spannender zu machen und den Design Prinzipien - wie in diesem Fall \ref {rule:2} "`Real Objects are more fun than buttons and menus"' und \ref{rule:2} "`Delight me in surprising ways"' . \\
Die Aufteilung zwischen den drei Hauptpunkten bleibt weiterhin bestehen und der Bereich der Dokumentation wird in einer beziehungsweise zwei weiteren Ebene untergliedert. \\
Es wurde diskutiert, ob die untergliederte Ebene Berechnung, siehe \ref{nav:doku}, lieber noch auf die erste Ebene für ein schnelleres Erreichen angeordnet werden soll. Die Vorteile sind klar in einer eindeutig besseren Performance zu sehen, allerdings würde die Übersichtlichkeit das ganze schwer zu spüren bekommen.
\\Als weitere mögliche Alternative besteht die Möglichkeit, dem Benutzer die Möglichkeit zu geben, seine wichtigsten Berechnungen auf der ersten Ebene hinzuzufügen. Aber auch hier haben wir uns gegen entschieden, da jeder Punkt innerhalb von maximal drei Interaktionen erreicht werden kann und die Übersichtlichkeit auf einem hohen Level davon profitiert.\\

Der Bereich Pflege besitzt ebenfalls ein Menü auf der ersten Ebene, wie bei \ref{nav:doku}, besitzt aber keine weitere Untergliederung. Der Bereich Probleme benötigte keine weitere Untergliederung, weswegen wir bei diesem Punkt ein weiteres Menü für nicht mehr nötig gehalten haben. Eine weitere Möglichkeit für die Navigation wäre zum Beispiel noch ein Kachelmenü gewesen, da dies aber eher typisch für Windows Phones ist, haben wir uns auch gegen diese Umsetzung entschieden.

\begin{figure}
	\centering
	\includegraphics[width=180px,height=\textheight,
keepaspectratio]{nav_doku}
	\includegraphics[width=180px,height=\textheight,
keepaspectratio]{nav_doku_Berechnungen}
	\caption{Navigationspfad Dokumentation}
	\label{nav:doku}
\end{figure}



\subsection{Pflege}
Bei der Pflege gehen wir genauer auf den \ref{care:ww} Wasserwechsel ein, da dieser die typische Eingabemaske, die Repräsentation der anschließend erhaltenen Informationen, Grundlagen für die Interaktion und auch einen Togglebutton beinhaltet. 
\\
Die Anordnung war aufgrund des Screen Design Standards aus \ref{sds} schon größtenteils Vorgegeben. Zunächst waren wir aber mit der Menge der Eingabefelder überfordert, weil wir anfangs noch zwei weitere Eingabefelder vorliegen hatten. Die Werte dieser Felder sind aber notwendigerweise mit der Erstöffnung anzugeben. Dadurch konnten wir diese Felder weglassen und dem Benutzer zum einen das Gefühl geben, dass er die schweren Berechnungen alleine durchführt und des weiteren auch noch redundante Informationen vermeiden. \\
Die größte Überlegung auf \ref{care:ww} wurde anhand der Anordnung der zwei Container angestellt. Da bei einer Eingabe durch den Nutzer bei \ref{care:ww:a} beide Felder verschoben werden müssten, haben wir uns für Variante \ref{care:ww:b} entschieden. 

\begin{figure}
	\begin{subfigure}[b]{0.5\textwidth}
		\includegraphics[width=\textwidth]{Wasserwechsel}
		\caption{Variante A}
		\label{care:ww:a}
	\end{subfigure}	
	\begin{subfigure}[b]{0.5\textwidth}
		\includegraphics[width=\textwidth]{wasserwechsel2}
		\caption{Variante B}
		\label{care:ww:b}
	\end{subfigure}	
	\caption{Pflege - Wasserwechsel}
	\label{care:ww}
\end{figure}

\subsection{Dokumentation}
Wichtige Bestandteile der Dokumentation fürs Aquarium sind die Berechnung des CO2- Gehalts, die Veränderung der Wasserwerte und das virtuelle Aquarium. Diese Teilpunkte können auch auf die anderen Berechnungen bezogen werden. 

\subsubsection{Virtuelles Aquarium}
Die Entscheidung ein virtuelles Aquarium zu erstellen, wurde deswegen getroffen, damit der Benutzer ein Element aus der wahren Welt hat. Zunächst war hier nur geringe Visualisierung und Unterstützung des Design Prinzips \ref{rule:2}, welche sich als Verwendung von realen Objekten in dem User Interface definiert, gegeben. Damit war \ref{doku:aq:c} 
als Gestaltungslösung zunächst festgelegt. Nachdem wir weitere Ideen gesucht hatten, haben wir uns auf eine neue Lösung fokussiert, welche ein neues Element mit der alten Gestaltungslösung \ref{doku:aq:c} kombiniert. 

Diese Lösung beinhaltet nun im oberen Bereich ein Foto, welches der Nutzer von seinem eigenen Aquarium aufnehmen muss und festen Werten. Wenn der Nutzer sich nun seine einzelnen Bestandteile anschauen möchte, kann dieser auf der Liste weiter nach unten swipen, um schließlich von der Ansicht \ref{doku:aq:a} zu der Ansicht \ref{doku:aq:c} zu kommen. Im Bereich \ref{doku:aq:c} kann der Benutzer des weiteren noch neue Objekte hinzufügen. Das Löschen funktioniert über das bekannte swipen nach links, in welcher Folge dann ein Button zum Löschen eingeblendet wird.\\
Die Werte des oberen Bereiches können selbstverständlich durch den Benutzer geändert werden, falls er sich ein neues Aquarium kaufen möchte oder auch einfach nur das Bild von seinem Aquarium aktualisieren möchte. Dies passiert über die in \ref{doku:aq:b} angezeigte Maske. Die einzelnen Eingaben laufen dabei über das zentrale Eingabefeld. \\
Alternativ hätte man diese Informationen auch durch längeres Betätigen einer Taste ändern können, allerdings ist die Änderung der vorgegebenen Datensätze nur sehr selten nötig.

\begin{figure}
	\begin{subfigure}[b]{0.3\textwidth}
		\includegraphics[width=\textwidth]{virtuellesAquarium}
		\caption{Übersicht}
		\label{doku:aq:a}
	\end{subfigure}	
	\begin{subfigure}[b]{0.3\textwidth}
		\includegraphics[width=\textwidth]{virtuellesAquariumEdit}
		\caption{Ändern der Grunddaten}\label{doku:aq:b}
	\end{subfigure}	
	\begin{subfigure}[b]{0.3\textwidth}
		\includegraphics[width=\textwidth]{virtuellesAquarium2}
		\caption{Bestandteile Aquarium}\label{doku:aq:c}
	\end{subfigure}	
	\caption{Dokumentation - Virtuelles Aquarium}
	\label{doku:va}
\end{figure}

\subsubsection{Wasserwerte}
Die Darstellung der Wasserwerte sind für uns bereits von Beginn an klar gewesen, daher kam es hier auch nicht zur Entwicklung von Alternativen. Eine bessere visuelle Darstellung gibt es vermutlich, aber bei diesem Punkt war uns die Übersichtlichkeit am wichtigsten, damit der Benutzer direkt erkennt, wo seine problematischen Werte liegen und er direkt für Besserung sorgen kann. Diese Übersichtlichkeit ist durch die farbliche Gestaltung gegeben, wie man in \ref{doku:ww} erkennt. 

\begin{figure}
	\centering
	\includegraphics[width=180px,height=\textheight,
keepaspectratio]{Wasserwerte}
	\caption{Dokumentation - Wasserwerte}
	\label{doku:ww}
\end{figure}

\subsubsection{CO2-Gehalt}
Wie bereits zu Abbildung \ref{care:ww} erläutert, bestand die Möglichkeit zwischen diesen beiden Gestaltungsmöglichkeiten. Zusätzlich zu der anderen Berechnung gibt es hier noch eine grafische Darstellung der Veränderung. Eigentlich präferieren wir \ref{doku:co2} in diesem Fall, die Gebrauchstauglichkeit des Systems wird aber aufgrund der Einheitlichkeit mehr durch die zweite Variante ermöglicht.  

\begin{figure}
	\begin{subfigure}[b]{0.5\textwidth}
		\includegraphics[width=\textwidth]{co2gehalt}
		\caption{Variante A}
		\label{doku:co2:a}
	\end{subfigure}	
	\begin{subfigure}[b]{0.5\textwidth}
		\includegraphics[width=\textwidth]{co2_gehalt_2}
		\caption{Variante B}
		\label{doku:co2:b}
	\end{subfigure}	
	\caption{Dokumentation - Berechnung - CO2-Gehalt}
	\label{doku:co2}
\end{figure}

\subsection{Probleme}
Im Bereich der Problembehandlung soll die direkte Kommunikation vom Kunden und dem Fachhändler bei Problemen mit dem Aquarium stattfinden. In der Übersicht \ref{probs} kann eine Problembeschreibung eingereicht werden und falls der Fachhandel keine Lösung findet, wird der ausgegraute Anrufen-Button freigeschaltet. Die Beschreibung wird aufgrund ihrer Länge nur in reduzierter Form angezeigt und die ausgegrauten Flächen skizzieren Anhänge. \\
Mit Hilfe des Stiftes kommt man zu einer Suche mit Auto-Vervollständigung, in welcher Aquariengeschäfte in der Nähe gesucht werden können, um diese als Kommunikationspartner festzulegen. \\
Auch hier ist wieder der Fokus auf eine möglichst simple und schnelle Verarbeitung gelegt, die Begrenzung des Textfeldes dient zum Beispiel dazu, dass dem Fachhandel keine unnötig langen Problembeschreibungen zugesendet werden. Durch die Verwendung der Kamera als weitere Möglichkeit der Dokumentation wird dem Benutzer ein weiterer Grund gegeben, dieses System als aktives Tool zu benutzen. 
		
\begin{figure}
	\centering
	\includegraphics[width=180px,height=\textheight,
keepaspectratio]{Probleme}
	\caption{Probleme - Übersicht}
	\label{probs}
\end{figure}

\chapter{Desktop Anwendung UI}

Als nächstes wollen wir den Prozess zur Erstellung des User Interface für die Desktop Anwendung dokumentieren. Begonnen haben wir mit einer Startansicht zur Suche und zum Auswählen eines Kunden. In Abbildung ~\ref{desktopUI:1} sieht man, dass wir oben auf der Seite Tabs erstellt haben, über die man zwischen verschiedenen Kunden herspringen kann. Diese Tabs können mit einem Klick auf das x geschlossen werden. Um die Daten eines neuen Kunden aufzurufen, muss auf auf den + Tab geklickt werden. Dort öffnet sich dann die Tabelle mit allen Kunden sowie eine Suche, um einen Kunden schneller finden zu können. Bei der Suche haben wir uns gegen einen Button zum Auslösen der Suche entschieden, da es komfortabler ist, wenn die Tabelle sich zur Echtzeit während der Texteingabe automatisch aktualisiert. Da die Tabelle sehr lang werden kann, haben wir hier bewusst zwei Farben für die Tabellenreihen gewählt, die sich von Reihe zu Reihe abwechseln, um eine bessere Übersicht haben zu können. Dieser der Teil der Anwendung wurde noch nicht im konzeptuellen Modell beachtet, da er in der Prozesshierarchie des Reengineering nicht aufgetaucht ist. Das schnelle Wechseln zwischen Kunden war uns aber sehr wichtig, sodass wir uns für diese Tab-Variante entschieden haben.

\begin{figure}[ht!]
\centering
\includegraphics[width=\linewidth]{1Startbildschirm}
\caption{Desktop UI: Startbildschirm}
\label{desktopUI:1}
\end{figure}

Das wurde allerdings zum Nachteil im nächsten Schritt. Denn jetzt sollte der erste im konzeptuellen Modell geplante Screen kommen. Dort haben wir nämlich ebenfalls Tabs für die oberste Hierarchie-Ebene vorgesehen und so gesehen auch für die zweite Ebene, nur dann als Drop-Down Liste, falls es weitere Unterelemente geben sollte. Das wurde beim designen der Seite zu einem Problem, da das ganze ziemlich verschachtelt wurde. Wir haben versucht es ein bisschen besser zu machen, indem wir die Tabs der obersten Ebene unterhalb der Box platziert haben. Somit waren zumindest schon mal nicht drei Menü-Leisten direkt untereinander. Das Ergebnis kann in Abbildung 
~\ref{desktopUI:0} betrachtet werden.

\begin{figure}[ht!]
\centering
\includegraphics[width=\linewidth]{0Version1}
\caption{Desktop UI: Verworfene Version}
\label{desktopUI:0}
\end{figure}

Da das ganze aber immer noch recht verschachtelt wirkte und wir nicht mit der Lösung zufrieden waren, haben wir uns dazu entschieden vom konzeptuellen Modell und vom Reengineering abzuweichen und eine alternative Lösung zu erstellen. Bei dieser Variante haben wir uns dann für das so genannte Accordion-Element entschieden. Dafür haben wir die zuvor verschiedenen Ebenen quasi gleichgestellt, denn jetzt gibt es keine Untermenüs mehr, sondern alle zuvor geplanten Screens werden als ein Item im Accordion-Element dargestellt. Dieses Element funktioniert so, dass alle Seiten untereinander aufgelistet sind und zu einem Zeitpunkt immer nur eine dieser Seiten geöffnet ist. Allerdings sieht man auch zu jedem Zeitpunkt die Verknüpfung zu den anderen Seiten und kann wie bei den Tabs schnell zwischen den verschiedenen Bereichen wechseln. Wenn man auf eine Verknüpfung zu einer anderen Seite klickt wird die aktuelle Seite eingefahren und die neue Seite ausgefahren. Diese Umsetzung war sehr zufriedenstellend, da es im Vergleich zu vorher viel übersichtlicher geworden ist. In Abbildung ~\ref{desktopUI:2} kann das Ergebnis sowie die erste Seite, den Kundeninformationen, angesehen werden. Auf der Kundeninformationen-Seite werden zuerst die allgemeinen Daten des Kunden angezeigt. Zusätzlich wird noch sein virtuelles Aquarium dargestellt sowie eine Tabelle mit allen Interaktionen mit dem Kunden. Dieses ``Logbuch'' war eine spontane Idee und wurde zuvor noch nicht aufgeführt. Die Darstellung der Fische, Pflanzen und dem Equipment haben wir in diesem Design noch recht einfach gehalten. Das könnte bei einer späteren Iteration oder während dem Entwicklungsprozess noch angepasst werden.

\begin{figure}[ht!]
\centering
\includegraphics[width=\linewidth]{2Kundeninformation}
\caption{Desktop UI: Neue Version mit Kundeninformationen}
\label{desktopUI:2}
\end{figure}

In Abbildung ~\ref{desktopUI:3} sieht man die nächste Seite, und zwar die Kaufberatung. Für die verschiedenen Kategorien gibt es jeweils einen Bereich. Jeder Bereich hat eine Liste mit passenden Kaufmöglichkeiten für den Kunden. Bei den Fischen und Pflanzen ist zusätzlich noch angegeben, wie der aktuelle Status diesbezüglich ist und wie viele Neuanschaffungen maximal möglich wären.

\begin{figure}[ht!]
\centering
\includegraphics[width=\linewidth]{3Kaufberatung}
\caption{Desktop UI: Kaufberatung}
\label{desktopUI:3}
\end{figure}

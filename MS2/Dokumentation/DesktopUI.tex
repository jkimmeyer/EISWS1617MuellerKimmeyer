\chapter{Desktop Anwendung UI}

Als nächstes wollen wir den Prozess zur Erstellung des User Interface für die Desktop Anwendung dokumentieren. Begonnen haben wir mit einer Startansicht zur Suche und zum Auswählen eines Kunden. In Abbildung ~\ref{desktopUI:1} sieht man, dass wir oben auf der Seite Tabs erstellt haben, über die man zwischen verschiedenen Kunden herspringen kann. Diese Tabs können mit einem Klick auf das x geschlossen werden. Um die Daten eines neuen Kunden aufzurufen, muss auf auf den + Tab geklickt werden. Dort öffnet sich dann die Tabelle mit allen Kunden sowie eine Suche, um einen Kunden schneller finden zu können. Bei der Suche haben wir uns gegen einen Button zum Auslösen der Suche entschieden, da es komfortabler ist, wenn die Tabelle sich zur Echtzeit während der Texteingabe automatisch aktualisiert. Da die Tabelle sehr lang werden kann, haben wir hier bewusst zwei Farben für die Tabellenreihen gewählt, die sich von Reihe zu Reihe abwechseln, um eine bessere Übersicht haben zu können. Dieser der Teil der Anwendung wurde noch nicht im konzeptuellen Modell beachtet, da er in der Prozesshierarchie des Reengineering nicht aufgetaucht ist. Das schnelle Wechseln zwischen Kunden war uns aber sehr wichtig, sodass wir uns für diese Tab-Variante entschieden haben.

\begin{figure}[ht!]
\centering
\includegraphics[width=\linewidth]{1Startbildschirm}
\caption{Desktop UI: Startbildschirm}
\label{desktopUI:1}
\end{figure}

Das wurde allerdings zum Nachteil im nächsten Schritt. Denn jetzt sollte der erste im konzeptuellen Modell geplante Screen kommen. Dort haben wir nämlich ebenfalls Tabs für die oberste Hierarchie-Ebene vorgesehen und so gesehen auch für die zweite Ebene, nur dann als Drop-Down Liste, falls es weitere Unterelemente geben sollte. Das wurde beim designen der Seite zu einem Problem, da das ganze ziemlich verschachtelt wurde. Wir haben versucht es ein bisschen besser zu machen, indem wir die Tabs der obersten Ebene unterhalb der Box platziert haben. Somit waren zumindest schon mal nicht drei Menü-Leisten direkt untereinander. Das Ergebnis kann in Abbildung 
~\ref{desktopUI:0} betrachtet werden.

\begin{figure}[ht!]
\centering
\includegraphics[width=\linewidth]{0Version1}
\caption{Desktop UI: Verworfene Version}
\label{desktopUI:0}
\end{figure}

Da das ganze aber immer noch recht verschachtelt wirkte und wir nicht mit der Lösung zufrieden waren, haben wir uns dazu entschieden vom konzeptuellen Modell und vom Reengineering abzuweichen und eine alternative Lösung zu erstellen. Bei dieser Variante haben wir uns dann für das so genannte Accordion-Element entschieden. Dafür haben wir die zuvor verschiedenen Ebenen quasi gleichgestellt, denn jetzt gibt es keine Untermenüs mehr, sondern alle zuvor geplanten Screens werden als ein Item im Accordion-Element dargestellt. Dieses Element funktioniert so, dass alle Seiten untereinander aufgelistet sind und zu einem Zeitpunkt immer nur eine dieser Seiten geöffnet ist. Allerdings sieht man auch zu jedem Zeitpunkt die Verknüpfung zu den anderen Seiten und kann wie bei den Tabs schnell zwischen den verschiedenen Bereichen wechseln. Wenn man auf eine Verknüpfung zu einer anderen Seite klickt wird die aktuelle Seite eingefahren und die neue Seite ausgefahren. Diese Umsetzung war sehr zufriedenstellend, da es im Vergleich zu vorher viel übersichtlicher geworden ist. In Abbildung ~\ref{desktopUI:2} kann das Ergebnis sowie die erste Seite, den Kundeninformationen, angesehen werden. Auf der Kundeninformationen-Seite werden zuerst die allgemeinen Daten des Kunden angezeigt. Zusätzlich wird noch sein virtuelles Aquarium dargestellt sowie eine Tabelle mit allen Interaktionen mit dem Kunden. Dieses ``Logbuch'' war eine spontane Idee und wurde zuvor noch nicht aufgeführt. Die Darstellung der Fische, Pflanzen und dem Equipment haben wir in diesem Design noch recht einfach gehalten. Das könnte bei einer späteren Iteration oder während dem Entwicklungsprozess noch angepasst werden.

\begin{figure}[ht!]
\centering
\includegraphics[width=\linewidth]{2Kundeninformation}
\caption{Desktop UI: Neue Version mit Kundeninformationen}
\label{desktopUI:2}
\end{figure}

In Abbildung ~\ref{desktopUI:3} sieht man die nächste Seite, und zwar die Kaufberatung. Für die verschiedenen Kategorien gibt es jeweils einen Bereich. Jeder Bereich hat eine Liste mit passenden Kaufmöglichkeiten für den Kunden. Bei den Fischen und Pflanzen ist zusätzlich noch angegeben, wie der aktuelle Status diesbezüglich ist und wie viele Neuanschaffungen maximal möglich wären.

\begin{figure}[ht!]
\centering
\includegraphics[width=\linewidth]{3Kaufberatung}
\caption{Desktop UI: Kaufberatung}
\label{desktopUI:3}
\end{figure}

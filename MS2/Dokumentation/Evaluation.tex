\chapter{Evaluation}
Bezüglich der Evaluation gibt Deborah Mayhew die Möglichkeit bei knappen Ressourcen den Evaluationsprozess zu überspringen und anhand der 10 Heuristiken von Jakob Nielsen\cite{nielsen:usability} zu evaluieren. \cite[249]{Mayhew:UEL}

Zunächst sollen diese noch kurz erläutert werden(frei übersetzt aus \cite{nielsen:usability}).
\begin{enumerate}
\item	Visibility of system status\\
Das System sollte den Nutzer immer darüber informieren, was gerade passiert. Das soll durch angemessenes Feedback in passender Zeit geschehen.
\item Match between system and the real world
Das System sollte sich an die Sprache des Benutzers anpassen. Das soll von den Wörtern, Sätzen und auch den Konzepten der Domäne passieren. Die Informationen sollen natürlich und logisch erscheinen.
\item User control and freedom
Der Benutzer wird Fehler in der Anwendung der Funktionen machen. Daher sollte ein Notausgang fest inbegriffen sein und die Möglichkeit Operationen rückgängig zu machen oder auch zu wiederholen.
\item Consistency and standards
Man sollte sich an den Standards orientieren und nicht von diesen abweichen. Der Benutzer soll sich nicht wundern, dass plötzlich etwas anderes passiert.
\item Error prevention
Besser als passende Fehlermeldung ist, wenn diese Fehler durch Prävention gar nicht erst auftreten können.
\item Recognition rather than recall
Objekte, Aktionen und Optionen sollen wahrnembar gemacht werden. Innerhalb der einzelnen Dialogteile soll der Benutzer sich keine Informationen merken müssen. Hinweise sollen immer auffindbar sein, wenn diese denn sinnvoll sind. 
\item Flexibility and efficiency of use
Abkürzungen - für den Anfänger nicht sichtbar - können die Interaktion für einen Experten des Systems beschleunigen. Das System soll sich an die jeweiligen Benutzer anpassen. 
\item Aesthetic and minimalist design
Nicht benötigte Informationen sollen bei jeder Möglichkeit vermieden werden. Durch jede weitere Informationen werden die relevanten Informationen weniger wichtig. 
\item Help users recognize, diagnose, and recover from errors
Fehlermeldung sollten konkret und leicht verständlich sein. Inhaltlich soll der Indikator für das Problem beschrieben werden und eine Lösung gegeben werden.
\item Help and documentation
Auch wenn es besser ist, wenn ein System ohne Dokumentation benutzt werden kann, kann es nötig sein, diese anzubieten. Diese Informationen sollen einfach zu finden sein, abhängig von der jeweiligen Aufgabe, auch hier wieder klare Formulierungen und nicht zu ausführlich werden.
\end{enumerate}

Eine Evaluation ist auch noch angesetzt, aber war für uns bis jetzt aufgrund des zeitlichen Rahmens leider nicht in dieser Phase zu realisieren. Da eine Verfälschung der Daten für uns nicht in Frage kommt, werden wir diese Phase bewusst in den nächsten Schritt schieben unabhängig von dem geforderten Artefakt im Meilenstein. 

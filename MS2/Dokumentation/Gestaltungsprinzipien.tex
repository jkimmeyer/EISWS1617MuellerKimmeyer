\chapter{Gestaltungsprinzipien}
Die Gestaltungsprinzipien sind eine weitere Grundlage für das Erstellen eines gebrauchstauglichen Designs. Es gibt eine äußerst hohe Zahl an Gestaltungsprinzipien, welche zum Beispiel auf der Plattform oder auch auf der Produktfamilie basieren. Ebenso kann Literatur, welche sich auf User-Interface Design-Prinzipien beziehen hinzugezogen werden. 

Da uns im Rahmen des Projektes nicht so viel Zeit zur Verfügung steht, ist unsere Recherche geringfügig ausgefallen und wir haben uns an den plattformbasierten Gestaltungsprinzipien von Android \cite{Android:designprinciples} und Windows orientiert. Da wir allerdings für Windowsanwendung keine spezielle Vorlage gefunden haben, haben wir weitere Aufwendungen für Gestaltungsprinzipien bezüglich der Windows Applikation unterlassen. Wir konzentrieren uns hier auf die für unser System wichtigsten 4 Android Prinzipien, die anderen sind im Anhang \ref{app:generaldesignprinciples} zu finden.

\section{Android - Mobile Anwendung}

\subsection{Delight me in surprising ways}\label{rule:1}
\begin{quote} "`A beautiful surface, a carefully-placed animation, or a well-timed sound effect is a joy to experience. Subtle effects contribute to a feeling of effortlessness and a sense that a powerful force is at hand."
\end{quote}

Dieser Punkt ist für unser System von besonderer Bedeutung, damit der Benutzer über eine längere Dauer Spaß an unserem System hat. Dadurch profitiert der Nutzer sowohl von seinem schönen Aquarium als auch vom schönen System.
 
\subsection{Real objects are more fun than buttons and menus}\label{rule:2}
\begin{quote} "`Allow people to directly touch and manipulate objects in your app. It reduces the cognitive effort needed to perform a task while making it more emotionally satisfying."\end{quote}

Reale Objekte sind in der Aquaristik viel anwendungsbezogener und eignen sich gerade aufgrund von sehr langen Begriffen. Daher ist die Umsetzung von Objekten anstatt von Buttons und Menüs von großer Bedeutung.
 
\subsection{Only show what I need when I need it}\label{rule:3}
\begin{quote} "`People get overwhelmed when they see too much at once. Break tasks and information into small, digestible chunks. Hide options that aren't essential at the moment, and teach people as they go."\end{quote}

Der Bereich der Aquaristik ist sehr komplex und soll deswegen möglichst einfach zu überblicken sein, gerade bei der Menge der Informationen ist das Ausblenden des öfteren sinnvoll.
 
\subsection{Do the heavy lifting for me}\label{rule:4}
\begin{quote} "`Make novices feel like experts by enabling them to do things they never thought they could. For example, shortcuts that combine multiple photo effects can make amateur photographs look amazing in only a few steps."\end{quote}

Die meisten Benutzer unseres Systems sind im Rahmen der Aquaristik auf Hilfe angewiesen, daher ist es besonders sinnvoll, diesem Prinzip stärkere Beachtung zu schenken.
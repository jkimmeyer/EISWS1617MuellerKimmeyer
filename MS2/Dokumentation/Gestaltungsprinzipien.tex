\chapter{Gestaltungsprinzipien}
Die Gestaltungsprinzipien sind eine weitere Grundlage für das Erstellen eines gebrauchstauglichen Designs. Es gibt eine äußerst hohe Zahl an Gestaltungsprinzipien, welche zum Beispiel auf der Plattform oder auch auf der Produktfamilie basieren. Ebenso kann Literatur, welche sich auf User-Interface Design-Prinzipien beziehen hinzugezogen werden. 

Da uns im Rahmen des Projektes nicht so viel Zeit zur Verfügung steht, ist unsere Recherche etwas geringfügiger ausgefallen und wir haben uns an den plattformbasierten Gestaltungsprinzipien von Android und Windows orientiert. Wir konzentrieren uns hier auf die für unser System wichtigsten 10 Prinzipien, weitere sind im Anhang zu finden.

\section{Android - Mobile Anwendung}

\subsection{Delight me in surprising ways}
\begin{quote} "`A beautiful surface, a carefully-placed animation, or a well-timed sound effect is a joy to experience. Subtle effects contribute to a feeling of effortlessness and a sense that a powerful force is at hand."
\end{quote}

Dieser Punkt ist für unser System von besonderer Bedeutung, damit der Benutzer über eine längere Dauer Spaß an unserem System hat. Dadurch profitiert der Nutzer sowohl von seinem schönen Aquarium als auch vom schönen System.
 
\subsection{Real objects are more fun than buttons and menus}
\begin{quote} "`Allow people to directly touch and manipulate objects in your app. It reduces the cognitive effort needed to perform a task while making it more emotionally satisfying."\end{quote}

Reale Objekte sind in der Aquaristik viel anwendungsbezogener und eignen sich gerade aufgrund von sehr langen Begriffen. Daher ist die Umsetzung von Objekten anstatt von Buttons und Menüs von großer Bedeutung.
 
\subsection{Only show what I need when I need it}
\begin{quote} "`People get overwhelmed when they see too much at once. Break tasks and information into small, digestible chunks. Hide options that aren't essential at the moment, and teach people as they go."\end{quote}

Der Bereich der Aquaristik ist sehr komplex und soll deswegen möglichst einfach zu überblicken sein, gerade bei der Menge der Informationen ist das Ausblenden des öfteren sinnvoll.
 
\subsection{Do the heavy lifting for me}
\begin{quote} "`Make novices feel like experts by enabling them to do things they never thought they could. For example, shortcuts that combine multiple photo effects can make amateur photographs look amazing in only a few steps."\end{quote}

Die meisten Benutzer unseres Systems sind im Rahmen der Aquaristik auf Hilfe angewiesen, daher ist es besonders sinnvoll, diesem Prinzip stärkere Beachtung zu schenken.


\section{Windows - Desktopanwendung}
Die User Interface Design Guidelines von Windows waren nicht sehr direkt aufzuwinden, daher haben wir uns an den Guidelines von Nielsen und Molich orientiert.

\subsection{Consistency and standards}
\begin{quote} Interface designers should ensure that both the graphic elements and terminology are maintained across similar platforms. For example, an icon that represents one category or concept should not represent a different concept when used on a different screen.\end{quote}
 
Da wir sowohl eine Desktopanwendung und einer Mobilen Anwendung entwickeln, ist dieser Punkt von wichtiger Bedeutung, da die Mitarbeiter so direkt auch ein Verständnis der Applikation der Kunden haben und auch andersherum.
 
\subsection{Error prevention}
\begin{quote} Whenever possible, design systems so that potential errors are kept to a minimum. Users do not like being called upon to detect and remedy problems, which may on occasion be beyond their level of expertise. Eliminating or flagging actions that may result in errors are two possible means of achieving error prevention.\end{quote}

Neben diesen Vorteilen ist für den Mitarbeiter im Fachhandel auch noch der Zeitdruck ein wichtiger Faktor, welcher durch die Vermeidung von Fehlern direkt zu schnellerer Bearbeitung und zufriederenen Kunden führt. 

\subsection{Flexibility and efficiency of use}
\begin{quote} With increased use comes the demand for less interactions that allow faster navigation. This can be achieved by using abbreviations, function keys, hidden commands and macro facilities. Users should be able to customize or tailor the interface to suit their needs so that frequent actions can be achieved through more convenient means.\end{quote}

Da die Mitarbeiter im Fachhandel dieses Tool sehr regelmäßig benutzen werden, hat dieser Punkt besondere Bedeutung, da die einfache Bedienbarkeit nicht mehr im Vordergrund liegt.

\subsection{Aesthetic and minimalist design}
\begin{quote} Keep clutter to a minimum. All unnecessary information competes for the user's limited attentional resources, which could inhibit user’s memory retrieval of relevant information. Therefore, the display must be reduced to only the necessary components for the current tasks, whilst providing clearly visible and unambiguous means of navigating to other content.\end{quote}

Ein weiterer wichtiger Punkt, der eigentlich auf jedes interaktive System bezogen werden sollte. Da die Mitarbeiter immer schnell auf den Kunden reagieren können sollen, ist hier ein besonderes Augenmerk auf eine optimale Darstellung gelegt.
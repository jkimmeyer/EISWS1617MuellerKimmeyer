\chapter{Konzeptuelles Modell}

Der Design-Prozess des Usability Engineering Lifecycles von Mayhew besteht aus drei aufeinanderfolgenden Teilprozessen: Conceptual Model Design, Screen Design Standards und Detailed UI Design. Im Folgenden werden wir zuerst den Prozess des Conceptual Model Designs dokumentieren.

\section{Produkt- oder prozessorientiert}

Zuerst muss abgewogen werden, ob das konzeptuelle Modell produkt- oder prozessorientiert sein wird. Das produktorientierte Modell passt zu Anwendungen, bei denen der Benutzer individuelle Produkte erstellt, benennt und speichert. Beispiele hierfür sind Microsoft Word, Excel und PowerPoint. In diesen Anwendungen werden verschiedene Dokumente (Textdateien, Tabellen, Präsentationen), also Produkte, vom Benutzer erstellt und bearbeitet. Das prozessorientierte Modell ist für Anwendungen, in denen keine klar identifizierbaren Produkte entstehen. In diesen Anwendungen liegt der Hauptteil darin, einen Prozess zu unterstützen. Es können zwar auch persönliche Daten gespeichert und empfangen werden, aber meistens ist es so, dass alle Benutzer Zugriff auf die gleichen Informationen haben. Es werden also keine individuellen Produkte wie zum Beispiel Textdokumente erstellt. Ein Beispiel für eine prozessorientierte Anwendung wäre ein Aufgaben-Verwaltungssystem. Die Zuordnung unserer Anwendung zu einem dieser Modelle ist also recht deutlich. Es werden zwar individuelle Aquarien-Daten vom Benutzer eingegeben, allerdings entsteht dadurch kein Produkt. Viel mehr dient es dazu, einen Prozess zu unterstützen. Und zwar den Prozess, sein Aquarium zu verwalten und die Qualität zu optimieren. Von daher benötigen wir für unsere Anwendung ein prozessorientiertes Modell. 

\section{Identifizierung der Prozesse}

Im nächsten Schritt würde eine genaue Identifizierung der Produkte bzw. Prozesse anstehen. Da wir aber das prozessorientierte Modell benutzen und somit Prozesse identifizieren müssen, können wir uns diesen Schritt sparen, da die Prozesse bereits beim Work Reengineering identifiziert wurden. Die Aufgaben-Hierarchie bzw. das Reengineered Task Organization Model können wir also als Grundlage für unser Modell nehmen.

\section{Design Regeln}

Als nächstes müssen Design-Regeln für diese Prozesse erstellt werden.  Es muss also definiert werden, wie jede Ebene aus dem Reengineered Task Organization Model visuell repräsentiert wird. Da wir zwei Anwendungen für verschiedene Plattformen entwickeln, muss hier unterschieden werden. Bei der Desktop Anwendung können wir uns stark an Mayhew halten. Hierzu betrachten wir die Hierarchie des Reengineered Task Organization Models. Die Elemente der obersten Ebene, also die Kundenberatung und Bearbeitung der Kundendaten, werden als Tabs dargestellt. Die zweite Ebene wird dann in einem Teilbereich der Anwendung als Untermenü dargestellt. Die Elemente sind dann, sofern sie noch weitere Unterpunkte haben, also Drop-Down Listen dargestellt und die darunterliegenden Elemente sind dann dementsprechend Items dieser Listen. Das Ergebnis als Modell kann in Abbildung ~\ref{kmodell1} betrachtet werden.
\\ \\
Bei der Modellierung der Android App gab es für uns zwei Möglichkeiten, die wir beide modelliert haben. Die erste Möglichkeit wäre eine App mit einem Seitenmenü. Dieses Seitenmenü kann über einen Button in der Navigationsleiste ausgefahren werden. Im Seitenmenü werden dann die Elemente der ersten Ebene als Überschriften dargestellt. Unter den jeweiligen Überschriften stehen dann die dazugehörenden Links zu den Elementen der zweiten Ebene. Wenn man dann auf so einen Link gegangen ist, bekommt man entweder den Inhalt der Seite angezeigt oder falls es noch weitere Unterpunkte gibt eine Liste mit weiteren Links. In Abbildung ~\ref{kmodell21} kann man das Modell mit dem Seitenmenü sehen und in Abbildung ~\ref{kmodell22} wird dann noch die App mit eingefahrenem Seitenmenü und einer weiteren Liste mit Links zu den verschiedenen Berechnungen gezeigt.
\\ \\
Die zweite Möglichkeit wäre eine App mit Tabs im unteren Bereich des Bildschirms. Dort könnten die drei Elemente der obersten Ebene mit passenden Icons platziert werden. Wenn man dann auf ein Tab geht kommt eine Liste mit den Unterseiten, also den Elementen der zweiten Ebene. Wenn man darüber dann zu einer Seite navigiert wird entweder der Inhalt angezeigt oder erneut eine Liste mit weiteren Unterseiten. In Abbildung ~\ref{kmodell31} sieht man die Tabs am unteren Bildschirmrand sowie die Liste mit weiteren Links der Dokumentation. In Abbildung ~\ref{kmodell32} sieht man, dass zu der Seite ``Berechnungen'' navigiert wurde. Dort werden jetzt wiederum Links zu Unterseiten angezeigt. Außerdem ist in der Navigationsleiste ein Zurück-Button erschienen, um zur vorherigen Seite zurückkehren zu können. Die Tabs bleiben auf jeder Seite am unteren Bildschirmrand, sodass zu jeder Zeit mit einem Klick zwischen den Tabs gewechselt werden kann.

\section{Ausarbeitung der Modelle}

\begin{figure}[htbp]
\centering
\includegraphics[width=\linewidth]{KModell1}
\caption{Konzeptuelles Modell: Desktop Anwendung}
\label{kmodell1}
\end{figure}

\begin{figure}[htbp]
\centering
\includegraphics[width=0.4\linewidth]{KModell2_1}
\caption{Konzeptuelles Modell: Android App - Seitenmenü 1}
\label{kmodell21}
\end{figure}

\begin{figure}[htbp]
\centering
\includegraphics[width=0.4\linewidth]{KModell2_2}
\caption{Konzeptuelles Modell: Android App - Seitenmenü 2}
\label{kmodell22}
\end{figure}

\begin{figure}[htbp]
\centering
\includegraphics[width=0.4\linewidth]{KModell3_1}
\caption{Konzeptuelles Modell: Android App - Tabs 1}
\label{kmodell31}
\end{figure}

\begin{figure}[htbp]
\centering
\includegraphics[width=0.4\linewidth]{KModell3_2}
\caption{Konzeptuelles Modell: Android App - Tabs 2}
\label{kmodell32}
\end{figure}
\chapter{Plattform Capabilities und Constraints}

Damit UI-Designer wissen können, welche Design-Möglichkeiten vorhanden sind und welche nicht, müssen die Plattform spezifischen Möglichkeiten und Einschränkungen aufgelistet werden. Die Umsetzung unseres Systems erfolgt als Android App für normale Benutzer und als Desktop Anwendung für Fachhandlungen. In den folgenden Tabellen werden wir jeweils für die Anwendungen die Möglichkeiten und Einschränkungen darstellen.

\begin{table}[htb]
\centering
\caption{Plattform: Android App}
\begingroup
\renewcommand{\arraystretch}{1.4} % Vertical Padding ändern
\begin{tabularx}{\linewidth}{%
|>{\raggedright\arraybackslash}X%
|>{\raggedright\arraybackslash}X%
|>{\raggedright\arraybackslash}X%
|>{\raggedright\arraybackslash}X%
|%
}
\hline
\textbf{Eigenschaft}   	& \textbf{Möglich}                        			& \textbf{Möglich mit zus. Aufwand} 	& \textbf{Nicht möglich}     		\\ \hline
Betriebssystem Version 	& 4.0.3 und höher                         			&                                   			& Darunterliegende Versionen 	\\ \hline
Display Größe          	& 4 Zoll und größer                        			&                                   			& Kleinere Größen            		\\ \hline
Eingabe-Geräte         	& Virtuelle Tastatur, Touchscreen         		& Physische Bluetooth Tastatur      	&                            			\\ \hline
Internetverbindung     	& Verbindung über WLAN, mobile Verbindung 	&                                   			&                            			\\ \hline
Farben                 	& Alle beliebigen Farben                  			&                                   			&                            			\\ \hline
Spezial-Effekte        	& 3D, Video, Audio                        			&                                   			&                            			\\ \hline
GUI Werkzeuge          	& Siehe Android Komponenten \autocite{Android:Komponenten}               		&                                   			&                           			\\ \hline
Energieversorgung      	& Begrenzte Akkulaufzeit, Netzbetrieb     		&                                   			&                            			\\ \hline
Multitasking           	& x                                       				&                                   			&                            			\\ \hline
Bit-Mapped-Display     	& x                                       				&                                   			&                            			\\ \hline
Windowing              	&                                         				&                                   			& x                          			\\ \hline
\end{tabularx}
\endgroup
\end{table}

\begin{table}[htb]
\centering
\caption{Plattform: Windows Desktop Anwendung}
\begingroup
\renewcommand{\arraystretch}{1.4} % Vertical Padding ändern
\begin{tabularx}{\linewidth}{%
|>{\raggedright\arraybackslash}X%
|>{\raggedright\arraybackslash}X%
|>{\raggedright\arraybackslash}X%
|>{\raggedright\arraybackslash}X%
|%
}
\hline
\textbf{Eigenschaft}   	& \textbf{Möglich}                        			& \textbf{Möglich mit zus. Aufwand} 	& \textbf{Nicht möglich}     		\\ \hline
Betriebssystem Version 	& 10                         						& 7, 8                                  		& Darunterliegende Versionen 	\\ \hline
Display Größe          	& 12 Zoll und größer                        			&                                   			& Kleinere Größen            		\\ \hline
Eingabe-Geräte         	& Physische Tastatur, Bildschirmtastatur, Maus	&       							&                            			\\ \hline
Internetverbindung     	& Verbindung über LAN oder WLAN 			&                                   			&                            			\\ \hline
Farben                 	& Alle beliebigen Farben                  			&                                   			&                            			\\ \hline
Spezial-Effekte        	& 3D, Video, Audio                        			&                                   			&                            			\\ \hline
GUI Werkzeuge          	& Siehe Java GUI Komponenten \autocite{Java:Komponenten}               		&                                   			&                           			\\ \hline
Energieversorgung      	& Netzbetrieb, begrenzte Akkulaufzeit (Laptops)	&                                   			&                            			\\ \hline
Multitasking           	& x                                       				&                                   			&                            			\\ \hline
Bit-Mapped-Display     	& x                                       				&                                   			&                            			\\ \hline
Windowing              	& x                                       				&                                   			&                          			\\ \hline
\end{tabularx}
\endgroup
\end{table}
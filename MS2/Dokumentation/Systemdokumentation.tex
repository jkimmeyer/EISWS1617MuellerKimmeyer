\chapter{Systemdokumentation}

\section{Iteration der Kommunikationsmodelle}

Im Konzept haben wir bereits ein deskriptives und ein präskripives Kommunikationsmodell präsentiert. Durch den Projektfortschritt können diese nochmal bearbeitet und verbessert werden. Diesen Prozess werden wir nachfolgend für die beiden Modelle dokumentieren.

\subsection{Deskriptives Modell}

Ein wichtiger Punkt bei der Verbesserung des Kommunikationsmodells war der zeitliche Ablauf. Dieser wurde in der ersten Version des Modells nicht beachtet. Deshalb war der erste Schritt, diesen Ablauf mit einer Numerierung der übermittelten Daten bzw. Aktionen deutlich zu machen. Der zweite wichtige Punkt war, dass alle Interaktionen als Aktionen formuliert waren und kein Fokus auf die übermittelten Daten gelegt wurde. Deshalb haben wir die einzelnen Interaktionen so umformuliert, dass möglichst nur noch die übermittelten Daten angegeben sind. Da wir aber nicht komplett auf Aktionen verzichten wollten, haben wir diese kursiv im Modell dargestellt, damit die Daten sich von diesen abheben können. Das neue Modell wird in Abbildung 9.1 dargestellt.

\begin{figure}[htbp]
\centering
\includegraphics[width=0.85\linewidth]{Kommunikationsdiagramm1}
\caption{Deskriptives Kommunikationsmodell}
\end{figure}

\subsection{Präskriptives Modell}

Beim präskriptiven Modell haben wir dementsprechend die gleichen Änderungen vorgenommen. Zusätzlich wurde noch die Benutzer-ID eingeführt, mit der der Fachhändler gezielt über das System auf die Daten des Kunden zugreifen kann. Diese Benutzer-ID wird nach dem ersten Benutzen des Systems an den Benutzer gegeben und dieser muss sie dann an den Fachhändler weiter geben. Das neue präskriptive Modell kann in Abbildung 9.2 angesehen werden.

\begin{figure}[htbp]
\centering
\includegraphics[width=0.85\linewidth]{Kommunikationsdiagramm2}
\caption{Präskriptives Kommunikationsmodell}
\end{figure}
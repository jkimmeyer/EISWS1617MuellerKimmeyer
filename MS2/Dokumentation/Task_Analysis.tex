\chapter{Kontextbezogene Aufgaben-Analyse}

Die "Contextual Task Analysis" hat das Ziel ein deskriptives Modell der aktuellen Aufgaben zu modellieren. Dazu müssen laut Mayhew zum Einen Hintergrundinformationen der zu automatisierenden Arbeit recherchiert werden. Zum anderen empfiehlt Mayhew Personen, welche in diesem Fall im Bereich der Aquaristik arbeiten oder Besitzer von Aquarien sind, zu interviewen und dadurch Informationen zum Arbeitsbereich und der Aufgabenbewältigung zu erhalten. Mit diesen Informationen als Grundlage soll anschließend ein deskriptives Aufgabenmodell erstellt werden. \\

In unserem Fall haben wir aufgrund von  vorhandenen Erfahrungen im Aquaristikbereich und begrenzter Zeit auf die Sammlung von Informationen durch Interviews verzichtet. \\ 

Die Aufgaben- Analyse wurde also mit einer Wiederbetrachtung der Anforderungen begonnen, um die Grenzen bzw. den Rahmen des Systems zu erkennen. \\

Anschließend wurden auf der vorhergegangen Formulierung der User-Profiles, in Kombination mit den Stakeholdern, die wichtigsten Rollen im System festgelegt. Dies waren für uns eindeutig die Aquariumbesitzer und die Mitarbeiter im Fachhandel. Zunächst wollten wir eine Spezifikation im Bereich der Aquarienbesitzer vornehmen, was wir aber für überflüssig gehalten haben, da die Unterschiede im Arbeitsumfeld bereits im Nutzungskontext dokumentiert wurde und der Fokus auf der Aufgabenerledigung liegt. Weil die Komplexität unseres Projektes  keine unübersichtlichen Größen erreicht hat, war es für uns möglich, die meisten als Brief-Use-Cases\footnote T zu formulieren. Besondere Beachtung wurde neben der Beschreibung auf die Trigger und Probleme gelegt, welche zusätzlich zu den üblichen Kriterien hinzugefügt wurden.  \\

Die Szenarios wurden auf Grundlage der gemachten Erfahrungen verfasst und betrachten dabei ein oder mehrere Use-Cases und beziehen ebenso die Nutzungskontextanalyse mit \ref ein.\\

Nachdem dann letztendlich die grundlegenden Benutzeraufgaben formuliert wurden, widmeten wir uns dem Entwurf des aktuellen "User Task Organization Model".\cite{Mayhew:UEL}

\section{Task-Szenarios}
\subsection{Task: Dokumentation der Wasserwerte}
\textbf{User:} Aquarienbesitzer\\

\textbf{Definition:} Der Aquarienbesitzer ist zu Hause und interessiert sich für die Wasserwerte in seinem Aquarium, da er sich für eine bestimmte neue Fischart interessiert. Um diese Fischart halten zu können, sind bestimmte Nährwerte im Wasser benötigt, damit diese Fische überleben.\\

\textbf{Task Flow:}
\begin{enumerate}
 	\item Toni, der Aquarienbesitzer, geht zum Fachhandel mit seiner Wasserprobe und lässt diese analysieren.
 	\item Der Fachhandel analysiert diese Wasserwerte, während Toni wieder nach Hause fährt und auf die Ergebnisse wartet.
 	\item Nach ein paar Tagen erhält Toni eine Antwort vom Fachhandel, wie denn seine Wasserwerte momentan aussehen. Dazu fährt Toni wieder zum Fachhandel und lässt sich  hier auch noch die tägliche Änderungsrate dank einer alten Analyse berechnen.
 	\item Toni kennt nun seine Wasserwerte und weiß, dass seine Wasserwerte für den neuen Fisch optimal sind und legt sich daraufhin diesen neuen Fisch zu.
	 \item Der neue Fisch lebt ohne Probleme im neuen Aquarium und passt zum Gesamtbild sehr gut dazu.
\end{enumerate}
\textbf{Task Closure:} Dieses Szenarios nahm 3 Tage in Anspruch, am ersten Tag bringt Toni die Wasserwerte zur Fachhandlung, am 2. Tag findet die Analyse statt und am 3. Tag holt Toni sich die Wasserwerte ab und kauft sich den neuen Fisch.\\

Um diese Aufgabe zu unterstützen, sollte das User Interface...
\begin{itemize}
  \item Die tägliche Änderungsrate der Nährwerte automatisch berechnen
  \item Die Dokumentation der Wasserwerte zu digitalisieren
\end{itemize}	


\subsection{Task: Problembehandlung}
\textbf{User:} Aquarienbesitzer\\

\textbf{Description:} Der Aquarienbesitzer Mike hat Probleme mit seinem Aquarium. Des öfteren sterben seine Fische, es haben sich viele Algen gebildet und die Pflanzen sehen nicht so gut aus, wie er es sich damals beim Kauf des Aquariums erhofft hatte. Er benötigt Hilfe von den Experten.\\

\textbf{Task Flow:}
\begin{enumerate}
\item Mike ruft bei der Fachhandlung Fressnapf an und fragt nach Hilfe bezüglich der Probleme mit seinem Aquarium. 
 \item Fressnapf teilt ihm mit, dass sie gerne eine Wasserprobe aus seinem Aquarium analysieren würden, um zu erkennen, wo das Problem denn liegen würde.
 \item Mike bringt die Probe zu Fressnapf, wo sie die Wasserprobe zum Analysieren an die Wissenschaftler weitergeben.
 \item Die Wissenschaftler analysieren einen Tag später diese Wasserprobe und geben die Ergebnisse an die Fachhandlung weiter.
 \item Fressnapf vereinbart einen neuen Termin mit Mike, dieser kommt vorbei und ihm wird anschließend mitgeteilt, dass in seinem Wasser eine Mangelversorgung durch Eisen vorliegt. 
 \item Mike kauft einen Eisendünger, welcher die Mangelversorgung ausgleicht und ein besseres Klima für die Wasserpflanzen gibt.
 \item Als eine Woche später immer noch Fische sterben, ruft Mike erneut in der Fachhandlung an. Da seine Wasserwerte aber optimal sind, können sie Mike so nicht weiterhelfen und schicken einen Mitarbeiter zu ihm nach Hause.
 \item Der Mitarbeiter betrachtet die Umstände des Aquariums, die Anzahl der Fische und weitere mögliche Problemquellen und kommt zum Resultat, dass die Anzahl und Art der Fische für die Größe des Beckens ein Problem ist und diese sich aufgrund dessen gegenseitig auffressen. Des Weiteren konnte der Mitarbeiter die verstärkte Algenbildung auf den Standpunkt in der direkten Sonne zurückführen.
 \item Mike stellt sein Aquarium um und verschenkt die Hälfte seiner Fische, damit ein optimales Verhältnis für seine Fische und Wasserpflanzen besteht.
\end{enumerate}

 \textbf{Task Closure:} Dieses Szenario benötigt zwei Wochen bis zur Auffindung aller Probleme. Die Analyse des ersten Problems dauert bereits 3 Tage. Die Feststellung, dass noch mehr Probleme vorliegen wurde aufgrund mangelnder Informationen nicht direkt erkannt und die Vollendung des Szenarios verlängert sich um mehr als eine weitere Woche.\\

Um diese Aufgabe zu unterstützen, sollte das User Interface...
\begin{itemize}
  \item Eine Möglichkeit geben, dass Aquarium anzuschauen, ohne das der Mitarbeiter vorbeifahren muss
  \item Die Übersicht der alten Wasserwerte bereits im Vorhinein dokumentieren und dem Fachhandel bei Bedarf anschauen
  \item Die Fische, Pflanzen und Lampen im Aquarium dem Fachhandel präsentieren, damit der Mitarbeiter des Fachhandels direkt eine Übersicht über alle Problemquellen hat
\end{itemize}


\subsection{Task: Wasserwechsel durchführen}

\textbf{User:} Aquarienbesitzer\\

\textbf{Description:} Zur guten Aquarienpflege gehört ein wöchentlicher Wasserwechsel. Da das Grundwasser an manchen Wohnorten abweichende Werte zu den Zielwerten des Aquariumwassers besitzt, ist eine gezielte Dosierung von Osmose- und Grundwasser selten zu vermeiden.\\

\textbf{Task Flow:}
\begin{enumerate}
 \item Mia wohnt in der Nähe eines Kanals, wo das Grundwasser aus dem Kanal abgepumpt wird, dadurch weichen die Wasserwerte des Grundwassers stark von den optimalen Werten für ihr Aquarium ab. Da ihr die Pflege ihres Aquariums aber trotzdem sehr wichtig ist, führt sie einen Wasserwechsel mit Berücksichtigung der Zielwerte durch
 \item Zunächst analysiert Mia die Werte ihres aktuellen Aquariumwassers und erhält Gesamt- und Karbonathärte sowie den pH-Wert.
 \item Mit Hilfe der Grundwasserwerte vom Versorger kann sie anschließend das benötigte Verhältnis des Osmose- und des Grundwasser unter Berücksichtigung der Menge des Wasserwechsels per Hand berechnen.
 \item Mia mischt das Grund- und Osmosewasser und führt den Wasserwechsel in ihrem Aquarium durch.
 \item Da Mia die Werte ihres Aquariums sorgfältig dokumentiert, ist das Eintragen des neuen Wasserwertes sehr einfach.
\end{enumerate}

\textbf{Task Closure:} Die Durchführung der Aufgabe dauert nur 40 Minuten. 10 Minuten für die Berechnung, 10 Minuten für die Analyse und noch 15 Minuten für die Durchführung der Wasserprobe und 5 weitere Minuten für das Eintragen der Dokumentation\\

Um diese Aufgabe zu unterstützen, sollte das User Interface...
\begin{itemize}
  \item Die Berechnung des Aquarienbesitzer durchführen
  \item An die Durchführung des Wasserwechsels erinnern
  \item Die Dokumentation der Wasserwerte übernehmen und die Durchführung eines Wasserwechsels bereits vor der Durchführung anzuzeigen
\end{itemize}


\subsection{Task: Empfehlen von Fischen und Pflanzen}
\textbf{User:} Mitarbeiter des Fachhandels\\

\textbf{Description:} Laura arbeitet schon seit längerem als Mitarbeiterin für Fressnapf im Bereich der Aquaristik. Sie berät die Kunden vor allem beim Kauf von Fischen und Pflanzen und berücksichtigt dabei immer die Wasserwerte und andere Umstände der Kunden.\\

\textbf{Task Flow:}
\begin{enumerate}
\item Die Meiers sind nach Ankunft der Ergebnisse ihrer Wasserprobe wieder zu Laura in den Laden gekommen. Laura soll ihnen nun Empfehlungen zu möglichen neuen Fischen und Pflanzen geben.
\item Laura schaut sich die Wasserwerte an und erkennt schnell, dass für das Wasser der Meiers sehr pflegeleichte und unanfällige Pflanzen gewählt werden müssen. 
\item Laura zeigt den Meiers daraufhin, die ihrer Meinung nach schönsten Wasserpflanzen in dieser Kategorie. 
\item Die Meiers interessieren sich nicht so sehr für die Wahl der Pflanzen und stimmen einem Kauf direkt zu, doch im Aquarium nebenan haben sie Fische gefunden, welche ihnen sehr gut gefallen.
\item Da Laura aber direkt erkennt, dass diese Fische für die Menge der Fische, welche die Meiers im Aquarium haben, nicht geeignet sind, rät sie die Meiers vom Kauf dieser Fische ab. Diese schauen sich kurz weiter um.
\item Laura wird währenddessen von einem anderen Kunden angesprochen, der Hilfe bei seinen Wasserpflanzen braucht. Da Laura ihm kurz hilft, muss Laura anschließend die Wasserwerte der Meiers erneut aufsuchen, damit sie wieder ihre Situation in Erinnerung hat.
\item Die Meiers wollen aber trotzdem gerne Fische haben, welche in diese Richtung gehen, da Fressnapf aber nur eine ähnliche Fischart besitzt, welche aber etwas andere Wasserwerte benötigt, rät Laura zur Zugabe von Natriumhydrogenkarbonat um die Karbonathärte anzuheben, da so alle Fischarten und Pflanzen gehalten werden können.
\item Die Meiers kaufen sich neue Fische, Pflanzen und ein Düngemittel mit Natriumhydrogenkarbonat dank der individuellen und qualitativ hochwertigen Beratung von Laura.
\end{enumerate}

\textbf{Task Closure:} Die Beratung der Meiers dauert 1,5 Stunden. Die Wasseranalyse hat bereits im Vorhinein ein paar Tage in Anspruch genommen. Die Beratung ist sehr umfassend und nimmt daher eine große Menge an Zeit in Anspruch.\\

Um diese Aufgaben zu unterstützen, sollte das User Interface...
\begin{itemize}
  \item Der Fachhandlung alle Informationen(Wasserwerte, Fische, Pflanzen) zum Aquarium bereitstellen für eine optimale Beratung
\end{itemize}



\section{List of User Tasks for Aquariumusers}
\begin{itemize}
\item Wasserprobe entnehmen
\item Wasserprobe an die Fachhandlung bringen
\item Wasserwechsel durchführen
\item Wasserwechselverhältnis berechnen
\item Probleme finden
\item Aquaristische Berechnungen durchführen
\item Aquarium düngen
\item Fische kaufen
\item Pflanzen kaufen
\item Dokumentation der Wasserwerte
\item Aquarium sauber halten
\end{itemize}


\section{List of User Tasks for Fachhandel-Mitarbeiter}
\begin{itemize}
\item Kunden beraten
\item Wasserproben analysieren
\item Probleme analysieren
\item Probleme lösen
\item Düngemittel empfehlen
\item Aquarium des Kunden kennen
\item Aquaristische Berechnungen durchführen
\end{itemize}



\chapter{Kontextbezogene Aufgaben-Analyse}

Die "Contextual Task Analysis" hat das Ziel ein deskriptives Modell der aktuellen Aufgaben zu modellieren. Dazu müssen laut Mayhew zum Einen Hintergrundinformationen der zu automatisierenden Arbeit recherchiert werden. Zum anderen empfiehlt Mayhew Personen, welche in diesem Fall im Bereich der Aquaristik arbeiten oder Besitzer von Aquarien sind, zu interviewen und dadurch Informationen zum Arbeitsbereich und der Aufgabenbewältigung zu erhalten. Mit diesen Informationen als Grundlage soll anschließend ein deskriptives Aufgabenmodell erstellt werden.

In unserem Fall haben wir aufgrund von  vorhandenen Erfahrungen im Aquaristikbereich und begrenzter Zeit auf die Sammlung von Informationen durch Interviews verzichtet. 

Die Aufgaben- Analyse wurde also mit einer Wiederbetrachtung der Anforderungen begonnen, um die Grenzen bzw. den Rahmen des Systems zu erkennen. 

Anschließend wurden auf der vorhergegangen Formulierung der User-Profiles, in Kombination mit den Stakeholdern, die wichtigsten Rollen im System festgelegt. Dies waren für uns eindeutig die Aquariumbesitzer und die Mitarbeiter im Fachhandel. Zunächst wollten wir eine Spezifikation im Bereich der Aquarienbesitzer vornehmen, was wir aber für überflüssig gehalten haben, da die Unterschiede im Arbeitsumfeld bereits im Nutzungskontext dokumentiert wurde und der Fokus auf der Aufgabenerledigung liegt. Weil die Komplexität unseres Projektes  keine unübersichtlichen Größen erreicht hat, war es für uns möglich, die meisten als Brief-Use-Cases\footnote zu formulieren. Besondere Beachtung wurde neben der Beschreibung auf die Trigger und Probleme gelegt, welche zusätzlich zu den üblichen Kriterien hinzugefügt wurden. 




Die Szenarios wurden auf Grundlage der gemachten Erfahrungen verfasst und betrachten dabei ein oder mehrere Use-Cases und beziehen ebenso die Nutzungskontextanalyse mit \ref ein.

Nachdem dann letztendlich die grundlegenden Benutzeraufgaben formuliert wurden, widmeten wir uns dem Entwurf des aktuellen "User Task Organization Model".

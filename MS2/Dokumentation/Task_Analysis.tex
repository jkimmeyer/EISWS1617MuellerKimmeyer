\chapter{Kontextbezogene Aufgaben-Analyse}\label{taskanalysis}

Die ``Contextual Task Analysis'' hat das Ziel ein deskriptives Modell der aktuellen Aufgaben zu modellieren. Dazu müssen laut Mayhew unter anderem Hintergrundinformationen der zu automatisierenden Arbeit recherchiert werden. Zum anderen empfiehlt Mayhew Personen, welche in diesem Fall im Bereich der Aquaristik arbeiten oder Besitzer von Aquarien sind, zu interviewen und dadurch Informationen zum Arbeitsbereich und der Aufgabenbewältigung zu erhalten. Mit diesen Informationen als Grundlage soll anschließend ein deskriptives Aufgabenmodell erstellt werden. \\

In unserem Fall haben wir aufgrund von  vorhandenen Erfahrungen im Aquaristikbereich und begrenzter Zeit auf die Sammlung von Informationen durch Interviews verzichtet. \\ 

Die Aufgaben- Analyse wurde also mit einer Wiederbetrachtung der Anforderungen begonnen, um die Grenzen bzw. den Rahmen des Systems zu erkennen. \\

Anschließend wurden auf der vorhergegangen Formulierung der User-Profiles, in Kombination mit den Stakeholdern, die wichtigsten Rollen im System festgelegt. Dies waren für uns eindeutig die Aquariumbesitzer und die Mitarbeiter im Fachhandel. Zunächst wollten wir eine Spezifikation im Bereich der Aquarienbesitzer vornehmen, was wir aber für überflüssig gehalten haben, da die Unterschiede im Arbeitsumfeld bereits im Nutzungskontext dokumentiert wurde und der Fokus auf der Aufgabenerledigung liegt. Weil die Komplexität unseres Projektes  keine unübersichtlichen Größen erreicht hat, war es für uns möglich, die Aufgaben als Brief-Use-Cases zu formulieren. Besondere Beachtung wurde neben der Beschreibung auf die Trigger und Probleme gelegt, welche zusätzlich zu den üblichen Kriterien hinzugefügt wurden, damit mögliche Potentiale für das zu entwickelnde System entdeckt werden. \\

\begin{table}[]
\centering
\caption{My caption}
\label{my-label}
\resizebox{\textwidth}{!}{%
\begin{tabular}{|p{4cm}|p{4cm}|p{6cm}|p{3cm}|p{4cm}|l}
\hline
\textbf{Actor} & \textbf{Goal} & \textbf{Brief Description} & \textbf{Trigger} & \textbf{Problem} \\ \hline
\multirow{7}{*}{Aquarienbesitzer} & Dokumentation der Wasserwerte & Der Besitzer eines Aquariums bringt eine Wasserprobe zum Fachhändler, welche diese analysiert und die Ergebnisse dem Besitzer zur permanenten Speicherung zurückgibt. & Probleme im Aquarium/ optimale Bedingungen sind dem Aquarianer wichtig & Dokumentation ist umständlich. Eintragen ist aufwendig \\ \cline{2-5} 
 & Dokumentation der Wasserwerte & Der Besitzer besitzt ein Tool zur Wasseranalyse und dokumentiert anschließend die Werte permanent. & Probleme im Aquarium/ optimale Bedingungen sind dem Aquarianer wichtig & Eintragen der Wasserwerte aufwendig. Papier ist nicht sehr wasserbeständig. Spritzwassergefahr!! \\ \cline{2-5} 
 & Düngung des Wassers & Basierend auf den Idealwerten der Nährstoffe und der aktuellen Wasserwerte wird die optimale Düngerdosierung berechnet und der Besitzer fügt die angemessene Menge an Düngemittel dem Wasser hinzu. & Es fehlt ein bestimmter Nährstoff im Wasser & Berechnung der Dosierung und das Kennen der Idealwerte sehr umständlich. Fehlendes Wissen \\ \cline{2-5} 
 & Fachhändler das Aquarium präsentieren & Der Besitzer möchte die einzelnen Komponenten, wie Fische, Pflanzen, Lampen und Boden dem Fachhandel präsentieren, damit er eine bessere Beratung vornehmen kann. & Kauf neuer Objekte/ Beratung durch Fachhandel/ Probleme im Aquarium & Fotos und Beschreibungen oft ungenau, unvollständig oder auch zu umständlich. \\ \cline{2-5} 
 & Zielgerichteten Wasserwechsel durchführen & Anhand der Werte des Leitungswasser wird das Verhältnis zwischen Osmose- und Leitungswasser für die richtige Zielmenge berechnet. Der Besitzer muss anschließend diesen WW durchführen. & Wöchentliche Aquariumpflege / Falsche Wasserwerte & Berechnung ist aufwendig. \\ \cline{2-5} 
 & Probleme behandeln & Bei Problemen die nicht auf die Wasserwerte zurückzuführen sind, muss der Aquarianer einen Mitarbeiter aus dem Fachhandel zu sich nach Hause bestellen. So kann dieser vor Ort die Umstände prüfen. & Problemfindung auf Entfernung nicht möglich & Die Fahrt zu einem Kunden nach Hause ist sehr kostenintensiv und zeitaufwendig. Termine sind oft nur in weiter Entferung zu erhalten. \\ \cline{2-5} 
 & Kauf von Objekten fürs Aquarium & Der Kunde kauft im Fachhandel neue Objekte fürs Aquarium, wie zum Beispiel Fische oder Wasserpflanzen. Dabei muss dieser darauf achten, dass die gekauften Objekte zu den übrigen Umständen passen. & Dem Aquarianer fehlt etwas im Aquarium & Der Kauf von Objekten passt oft nicht zu dem individuellen Aquarium. \\ \hline
\multirow{4}{*}{Fachhandel} & Analyse der Wasserwerte & Der Kunde(Aquarienbesitzer) bringt die Probe zum Fachhandel, der Fachhandel ist anschließend dafür verantwortlich, die Proben zu analysieren und die Werte an den Kunden weiterzugeben. & Kunde möchte eine Wasseranalyse & Zeitliche Dauer, der Kunde muss oft zweimal zum Fachhandel \\ \cline{2-5} 
 & Beratung des Kunden aufgrund von Problemen & Fachhandel schaut sich die Wasserwerte der Kunden an und reagiert daraufhin mit Empfehlung des optimalen Düngemittels. Die Dosierung übernimmt der Kunde selbst. & Schlechte Wasserwerte im Aquarium des Kunden & Dokumentation der Wasserwerte und Empfehlungen auf Papier geht oft verloren. Verlangt Anwesenheit des Kundens im Geschäft \\ \cline{2-5} 
 & Optimale Beratung des Kunden & Empfehlungen von Fischen und Pflanzen anhand der analysierten Wasserwerte. Der Kunde erhält so eine individuelle Analyse und der Fachhändler kann entsprechend der Wasserwerte auf Probleme reagieren. & Kunde möchte beraten werden & Individuelle Aquarienübersicht der Kunden schwierig zu überblicken, da die Kunden oft wechseln und die Wasserwerte nur von einmaliger Messung vorliegen \\ \cline{2-5} 
 & Problemanalyse beim Kunden & Anhand der Wasserwerte oder auch Besuch bei einem Kunden kann das Aquarium auf alle möglichen Einflüsse, die zu Problemen führen, überprüft werden. & Problemfindung auf Entfernung nicht möglich & Die Fahrt zu einem Kunden nach Hause ist sehr kostenintensiv und zeitaufwendig. Termine sind oft nur in weiter Entferung zu erhalten. \\ \hline
\end{tabular}%
}
\end{table}

Die Szenarios wurden auf Grundlage der gemachten Erfahrungen von Johannes verfasst und betrachten dabei einen oder mehrere Use-Cases und beziehen ebenso die Nutzungskontextanalyse mit ein. Die zwei wichtigsten Task-Szenarios, welche sich ebenso sehr auf unsere Alleinstellungsmerkmale fokussieren, sind hier aufgeführt. Weitere Szenarien sind im Anhang \ref{app:taskscenarios} zu finden.\\

\section{Task-Szenarios}
\subsection{Task: Problembehandlung}
\textbf{User:} Aquarienbesitzer\\

\textbf{Description:} Der Aquarienbesitzer Mike hat Probleme mit seinem Aquarium. Des öfteren sterben seine Fische, es haben sich viele Algen gebildet und die Pflanzen sehen nicht so gut aus, wie er es sich damals beim Kauf des Aquariums erhofft hatte. Er benötigt Hilfe von den Experten.\\

\textbf{Task Flow:}
\begin{enumerate}
\item Mike ruft bei der Fachhandlung Fressnapf an und fragt nach Hilfe bezüglich der Probleme mit seinem Aquarium. 
 \item Fressnapf teilt ihm mit, dass sie gerne eine Wasserprobe aus seinem Aquarium analysieren würden, um zu erkennen, wo das Problem denn liegen würde.
 \item Mike bringt die Probe zu Fressnapf, wo sie die Wasserprobe zum Analysieren an die Wissenschaftler weitergeben.
 \item Die Wissenschaftler analysieren einen Tag später diese Wasserprobe und geben die Ergebnisse an die Fachhandlung weiter.
 \item Fressnapf vereinbart einen neuen Termin mit Mike, dieser kommt vorbei und ihm wird anschließend mitgeteilt, dass in seinem Wasser eine Mangelversorgung durch Eisen vorliegt. 
 \item Mike kauft einen Eisendünger, welcher die Mangelversorgung ausgleicht und ein besseres Klima für die Wasserpflanzen gibt.
 \item Als eine Woche später immer noch Fische sterben, ruft Mike erneut in der Fachhandlung an. Da seine Wasserwerte aber optimal sind, können sie Mike so nicht weiterhelfen und schicken einen Mitarbeiter zu ihm nach Hause.
 \item Der Mitarbeiter betrachtet die Umstände des Aquariums, die Anzahl der Fische und weitere mögliche Problemquellen und kommt zum Resultat, dass die Anzahl und Art der Fische für die Größe des Beckens ein Problem ist und diese sich aufgrund dessen gegenseitig auffressen. Des Weiteren konnte der Mitarbeiter die verstärkte Algenbildung auf den Standpunkt in der direkten Sonne zurückführen.
 \item Mike stellt sein Aquarium um und verschenkt die Hälfte seiner Fische, damit ein optimales Verhältnis für seine Fische und Wasserpflanzen besteht.
\end{enumerate}

 \textbf{Task Closure:} Dieses Szenario benötigt zwei Wochen bis zur Auffindung aller Probleme. Die Analyse des ersten Problems dauert bereits 3 Tage. Die Feststellung, dass noch mehr Probleme vorliegen wurde aufgrund mangelnder Informationen nicht direkt erkannt und die Vollendung des Szenarios verlängert sich um mehr als eine weitere Woche.\\

Um diese Aufgabe zu unterstützen, sollte das User Interface...
\begin{itemize}
  \item Eine Möglichkeit geben, dass Aquarium anzuschauen, ohne das der Mitarbeiter vorbeifahren muss
  \item Die Übersicht der alten Wasserwerte bereits im Vorhinein dokumentieren und dem Fachhandel bei Bedarf anschauen
  \item Die Fische, Pflanzen und Lampen im Aquarium dem Fachhandel präsentieren, damit der Mitarbeiter des Fachhandels direkt eine Übersicht über alle Problemquellen hat
\end{itemize}

\subsection{Task: Empfehlen von Fischen und Pflanzen}
\textbf{User:} Mitarbeiter des Fachhandels\\

\textbf{Description:} Laura arbeitet schon seit längerem als Mitarbeiterin für Fressnapf im Bereich der Aquaristik. Sie berät die Kunden vor allem beim Kauf von Fischen und Pflanzen und berücksichtigt dabei immer die Wasserwerte und andere Umstände der Kunden.\\

\textbf{Task Flow:}
\begin{enumerate}
\item Die Meiers sind nach Ankunft der Ergebnisse ihrer Wasserprobe wieder zu Laura in den Laden gekommen. Laura soll ihnen nun Empfehlungen zu möglichen neuen Fischen und Pflanzen geben.
\item Laura schaut sich die Wasserwerte an und erkennt schnell, dass für das Wasser der Meiers sehr pflegeleichte und unanfällige Pflanzen gewählt werden müssen. 
\item Laura zeigt den Meiers daraufhin, die ihrer Meinung nach schönsten Wasserpflanzen in dieser Kategorie. 
\item Die Meiers interessieren sich nicht so sehr für die Wahl der Pflanzen und stimmen einem Kauf direkt zu, doch im Aquarium nebenan haben sie Fische gefunden, welche ihnen sehr gut gefallen.
\item Da Laura aber direkt erkennt, dass diese Fische für die Menge der Fische, welche die Meiers im Aquarium haben, nicht geeignet sind, rät sie die Meiers vom Kauf dieser Fische ab. Diese schauen sich kurz weiter um.
\item Laura wird währenddessen von einem anderen Kunden angesprochen, der Hilfe bei seinen Wasserpflanzen braucht. Da Laura ihm kurz hilft, muss Laura anschließend die Wasserwerte der Meiers erneut aufsuchen, damit sie wieder ihre Situation in Erinnerung hat.
\item Die Meiers wollen aber trotzdem gerne Fische haben, welche in diese Richtung gehen, da Fressnapf aber nur eine ähnliche Fischart besitzt, welche aber etwas andere Wasserwerte benötigt, rät Laura zur Zugabe von Natriumhydrogenkarbonat um die Karbonathärte anzuheben, da so alle Fischarten und Pflanzen gehalten werden können.
\item Die Meiers kaufen sich neue Fische, Pflanzen und ein Düngemittel mit Natriumhydrogenkarbonat dank der individuellen und qualitativ hochwertigen Beratung von Laura.
\end{enumerate}

\textbf{Task Closure:} Die Beratung der Meiers dauert 1,5 Stunden. Die Wasseranalyse hat bereits im Vorhinein ein paar Tage in Anspruch genommen. Die Beratung ist sehr umfassend und nimmt daher eine große Menge an Zeit in Anspruch.\\

Um diese Aufgaben zu unterstützen, sollte das User Interface...
\begin{itemize}
  \item Der Fachhandlung alle Informationen(Wasserwerte, Fische, Pflanzen) zum Aquarium auf einen Blick bereitstellen für eine optimale Beratung.
  \item Das schnelle Wechseln zwischen den einzelnen Informationen zu den Kunden ermöglichen.
\end{itemize}

Die grundlegenden Nutzeraufgaben haben wir erneut in die Aufgaben für Aquarium-Besitzer und Fachhandel-Mitarbeiter aufgeteilt. Bei den Aufgaben sind Gemeinsamkeiten zwischen beiden Gruppierungen zu erkennen, welche aber davon abhängig durchgeführt werden, wie viel die Erfahrung die einzelnen Personen letztendlich haben.

\section{List of User Tasks for Aquarium-Besitzer}
\begin{itemize}
\item Wasserprobe entnehmen
\item Wasserprobe an die Fachhandlung bringen
\item Wasserwechsel durchführen
\item Wasserwechselverhältnis berechnen
\item Probleme finden
\item Aquaristische Berechnungen durchführen
\item Aquarium düngen
\item Fische kaufen
\item Pflanzen kaufen
\item Dokumentation der Wasserwerte
\item Aquarium sauber halten
\end{itemize}

\section{List of User Tasks for Fachhandel-Mitarbeiter}
\begin{itemize}
\item Kunden beraten
\item Wasserproben analysieren
\item Probleme analysieren
\item Probleme lösen
\item Düngemittel empfehlen
\item Aquarium des Kunden kennen
\item Aquaristische Berechnungen durchführen
\end{itemize}

\section{User Task Organization Models}\label{utd}
Letztendlich widmeten wir uns dem Entwurf des aktuellen "User Task Organization Model".\cite{Mayhew:UEL}
Hier teilten wir erneut zwischen der Gruppe der Aquarienverwaltung für den Aquarium-Besitzer und der Kundenverwaltung für den Fachhandel-Mitarbeiter. Die Modelle sind in einer Top-Down Struktur realisiert worden und gliedert sich nach Hauptaspekt-Teilaspekt-Unteraspekt. 

\begin{figure}
	\centering
	\includegraphics[width=\linewidth,height=\textheight,
keepaspectratio]{utd_aqverwaltung}
	\caption{User Task Diagramm: Aquariumverwaltung}
	\label{utd_aqverwaltung}
\end{figure}

\begin{figure}
	\centering
	\includegraphics[width=\linewidth,height=\textheight,
keepaspectratio, angle=90]{utd_kuverwaltung}
	\caption{User Task Diagramm: Kundenverwaltung}
	\label{utd_kuverwaltung}
\end{figure}


\chapter{Usability Goals}

Auf Basis der zuvor erledigten Arbeitsschritte werden nun Gebrauchstauglichkeit-Ziele ermittelt. Spezifische Usability Goals helfen dabei, die Gestaltung der Benutzeroberfläche zu fokussieren, indem den Designern etwas Konkretes vorgelegt wird, wonach sie sich richten können. In Folgendem werden wir den Prozess zur Erstellung dieser Ziele dokumentieren und das Ergebnis in einer von Mayhew vorgeschlagenen Form darstellen.

\section{Auswertung der User Profiles und Task Analyse}

Der erste Schritt war die Auswertung der User Profiles und das Notieren von wichtigen Erkenntnissen bzgl. der Gebrauchstauglichkeits-Ziele. Hier sind wir vor allem zu dem Ergebnis gekommen, dass das Design einfach gehalten werden muss, um sowohl die Anwendung als auch die Thematik leicht verstehen zu können, da das System vermutlich hauptsächlich von Anfängern benutzt wird und da diese am meisten Probleme haben werden und somit auch am meisten auf das System und die Beratung angewiesen sind. Für die Fachhändler war am wichtigsten, schnell zwischen den Übersichten der Kunden wechseln zu können und bei Ablenkungen schnell wieder die Orientierung zu finden. Zusätzlich zu den User Profiles wurde dann noch die Task Analyse betrachtet und auch hier Informationen für die Usability Goals rausgschrieben.

\section{Qualitative Usability Goals}

Im nächsten Schritt wurden anhand der Ergebnisse aus der Auswertung der User Profiles und der Task Analyse die genauen qualitativen Usability Goals formuliert und diese priorisiert. Im Folgenden werden wir zuerst die Bedeutungen der Prioritäten nach Mayhew erläutern und danach die qualitativen Usability Goals auflisten.

\begin{itemize}
  \item 1 = Wird für die Veröffentlichung benötigt
  \item 2 = Wichtig, wenn die Erreichung nicht übermäßig teuer oder zeitrauben ist
  \item 3 = Wünschenswert, aber nur, wenn niedrige Kosten
\end{itemize}	

\subsection{Formulierung der Ziele}

\textbf{Z1: Das Eintragen der Werte in die Berechnungen muss schnell gehen}
\\ \\
Priorität: 1
\\ \\
Bei der ersten Anwendung der jeweiligen Berechnungen muss der Benutzer in jedes Feld den entsprechenden Wert eintragen. Danach könnte dieser Vorgang allerdings durch intelligente Vor-Ausfüllung der Felder optimiert werden. Es könnten zum Beispiel die Werte der letzten Berechnung in den Feldern vorausgefüllt sein, da diese sich vermutlich nicht besonders viel von den neuen Werten unterscheiden. Somit kann der Benutzer die Werte, die noch mit den älteren Werten übereinstimmen, überspringen und spart somit an Zeit.
\\ \\
\textbf{Z2: Der Nutzer muss dem Fachhandel sein Aquarium in möglichst kurzer Zeit darstellen können}
\\ \\
Priorität: 1
\\ \\
Das Eintragen seiner Aquarium Daten sollte nicht zu lange dauern. Bei mangelnder Beschreibung des geforderten Wert könnte es zum Beispiel sein, dass der Benutzer erst einmal recherchieren muss, wo er die geforderte Information findet. Dies könnte das Design abnehmen, indem auf Anhieb klar wird, welche Information benötigt wird und wo diese zu finden ist.
\\ \\
\textbf{Z3: Das Design muss Mitarbeiter im Fachhandel unterstützen, die oft abgelenkt werden}
\\ \\
Priorität: 1
\\ \\
Im Fachhandel kommt es oft vor, dass ein Mitarbeiter z.B. von einem Kunden oder von einem anderen Mitarbeiter abgelenkt wird. Das Design muss dafür sorgen, dass der Benutzer direkt wieder weiß wo er zu dem Zeitpunkt war und was er gemacht hat, als er abgelenkt wurde. Dies könnte durch das Anzeigen von vielen Kontextinformationen umgesetzt werden.
\\ \\
\textbf{Z4: Das Design muss Mitarbeitern im Fachhandel ermöglichen, schnell zwischen den Kontextinformationen zu den einzelnen Kunden zu wechseln}
\\ \\
Priorität: 1
\\ \\
Im Fachhandel kann es beispielsweise vorkommen, dass der Fachhändler einen Kunden bedient und zu dem Zeitpunkt einen Anruf eines anderen Kunden bekommt. Jetzt sollte der Fachhändler die Möglichkeit haben, schnell von den Informationen des einen Kunden zum anderen Kunden wechseln zu können und anschließend wieder schnell zu den Informationen des ersten Kunden zurück wechseln zu können.
\\ \\
\textbf{Z5: Das Design sollte selbsterklärend und leicht zu lernen sein}
\\ \\
Priorität: 1
\\ \\
Da die Thematik an sich schon nicht einfach ist, sollte das Design nicht noch zusätzlich kompliziert aufgebaut sein. Dazu kommt, dass vermutlich ein Großteil der Nutzer Anfänger sein werden. Es sollte für den Benutzer also selbsterklärend und leicht zu lernen sein. 
\\ \\
\textbf{Z6: Der Aquariumbesitzer sollte die graphische Darstellung seine Wasserwerte schnell verstehen}
\\ \\
Priorität: 1
\\ \\
Da es sich vor allem beim zeitlichen Verlauf der Wasserwerte anbietet eine graphische Darstellung zu nehmen, könnte es sein, dass diese nicht auf Anhieb verstanden wird. Deshalb sollte es eine einfache Darstellung zum Beispiel in Form eines Balkendiagramms und keine unnötig komplizierten Darstellungen geben.
\\ \\
\textbf{Z7: Die zum Kunden gehörenden Daten sollten vom Mitarbeiter innerhalb kurzer Zeit und mit sehr hoher Trefferwahrscheinlichkeit gefunden werden}
\\ \\
Priorität: 1
\\ \\
Da die Kundendatenbank einer Fachhandlung sehr groß werden kann, muss gewährleistet werden, dass der Mitarbeiter schnell und präzise die passenden Daten des vor ihm stehenden Kunden aufrufen kann. Es muss also erstens eine Suche vorhanden sein und zweitens ein optimales Design der Suchergebnisse, sodass der Mitarbeiter auch bei Kunden mit gleichem Namen auf dem ersten Blick den passenden Kunden finden kann, indem er zum Beispiel nach einer zusätzlichen Information wie dem Geburtsdatum fragt und somit in den Suchergebnissen diesen Kunden auswählen kann.
\\ \\
\textbf{Z8: Der Mitarbeiter im Fachhandel sollte die grafische Darstellung der Wasserwerte der Kunden schnell verstehen}
\\ \\
Priorität: 2
\\ \\
Hier gilt das gleiche wie bei Z6, allerdings hat dieses Ziel eine niedrigere Priorität, da der Mitarbeiter in der Regel Erfahrung mit den Wasserwerten hat und somit schneller versteht, was gemeint ist. Da die grafische Darstellung aber sowieso für den Benutzer einfach sein soll, ist es gleichzeitig auch für den Fachhändler gegeben.
\\ \\
\textbf{Z9: Das Design muss Aquariumbesitzer unterstützen, die oft abgelenkt werden}
\\ \\
Priorität: 2
\\ \\
Dieses Ziel ist ähnlich wie Z3, allerdings geht es hier um die Aquariumbesitzer, die eine andere Anwendung benutzen, wie die Fachhändler. Aquariumbesitzer können zum Beispiel durch ihre neugierigen Kinder oder eingehende Telefonate oder Ähnliches abgelenkt werden. Dieses Ziel hat allerdings eine niedrigere Priorität, da die Ablenkung bei Aquariumbesitzern wahrscheinlich nicht so hoch ist wie in einer Fachhandlung. Trotzdem ist es ein wichtiges Ziel und sollte möglichst umgesetzt werden.
\\ \\
\textbf{Z10: Bei einem Gespräch mit dem Kunden ohne direkte Anwesenheit, soll der Mitarbeiter durch die Problemanalyse geführt werden, damit alle möglichen Quellen überprüft werden}
\\ \\
Priorität: 3
\\ \\
Dieses Ziel dient dazu, dem Mitarbeiter eine Vereinfachung der Problemanalyse darzustellen. Da der Mitarbeiter aber gut ausgebildet sein sollte und alle möglichen Fehlerquellen auch im Kopf haben sollte, ist dieses Ziel nur wünschenswert, falls es nicht zu viel Aufwand ist. Es würde zumindest verhindern, dass etwas vergessen wird.
\\ \\
\textbf{Z11: Für Experten sollte es einen extra Modus geben, damit diese nicht mit zu vielen Informationen, die sie schon kennen, in Berührung kommen}
\\ \\
Priorität: 3
\\ \\
Da das System hauptsächlich von Anfängern benutzt wird, ist es eins unserer Ziele die Benutzung für diese so gut wie möglich zu vereinfachen und deshalb werden u.a. viele Informationen zu den geforderten Daten und Berechnungen usw. dargestellt. Da diese Informationen aber für Experten unnötig wären und eventuell stören würden, würde sich hier ein extra Modus anbieten. Dieses Ziel hat aber nur eine niedrige Priorität, da die Zielgruppe recht klein ist und der Aufwand vermutlich recht hoch wäre.

\section{Quantitative Usability Goals}

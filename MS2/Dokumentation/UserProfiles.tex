\chapter{User Profiles}

Um die allgemeinen Anforderungen einer Kategorie von Benutzern in Bezug auf das User Interface ermitteln zu können, haben wir nachfolgend zunächst User Profiles für die verschiedenen Benutzer unseres Systems aufgestellt. Wir unterscheiden unsere Benutzer erst nach dem Alter und dann nach Erfahrung im Themengebiet, wobei wir die Erfahrung bei Kindern und Jugendlichen außen vor lassen, da diese in der Regel nicht so viel Erfahrung in diesem Bereich haben. Zusätzlich zu den normalen Benutzern gibt es dann noch ein User Profile für die Fachhändler. Zur besseren Übersicht sind nachfolgend nur die User Profiles ``Erwachsener - Aquarium Neuling'' und ``Fachhändler'' dargestellt und der Rest befindet sich im Anhang. Bevor es mit den User Profiles losgeht, ist hier eine Übersicht der Unterteilungen:

\begin{itemize}
\item Kind/Jugendlicher
\item Erwachsener - Aquarium Anfänger
\item Erwachsener - Aquarium Fortgeschritten
\item Erwachsener - Aquarium Experte
\item Rentner - Aquarium Neuling
\item Rentner - Aquarium Fortgeschritten
\item Rentner - Aquarium Experte
\item Fachhändler
\end{itemize}

\renewcommand{\title}{Erwachsener - Aquarium Anfänger}
\section{\title}

Diese Benutzergruppe ist die größte Gruppe, da Anfänger vermutlich die meisten Probleme haben werden und dementsprechend auch am meisten auf das System und die Beratung angewiesen sind. 

\begin{filecontents}{user_profile_2.auto}
\begingroup
\renewcommand{\arraystretch}{1.4} % Vertical Padding ändern
\begin{longtable}{%
|>{\raggedright\arraybackslash}X%
|>{\raggedright\arraybackslash}X%
|%
}
\caption{\title}	\\
\multicolumn{1}{c}{\textbf{Merkmal}}   	& \multicolumn{1}{c}{\textbf{Merkmalsausprägung}}		\\ \hline
\endfirsthead
\caption{\title - Fortsetzung}	\\
\multicolumn{1}{c}{\textbf{Merkmal}}   	& \multicolumn{1}{c}{\textbf{Merkmalsausprägung}}		\\ \hline	
\endhead
\hline
\endfoot
1. Demographisch 						& 								                         	\\
			          					& 								                       	\\
Alter			         					& 18 - 67										      \\
Geschlecht		    					& Männlich / Weiblich								\\
Wohnort                 					& Deutschland							                  	\\
Sozial-ökonomischer Status				& - Kein Beruf / Ausbildung / Studium im aquaristischen oder zoologischen Bereich   \\
									& - Variables Einkommen	                        			\\ \hline
2. Berufserfahrung						& Kurze bis lange Berufserfahrung, allerdings nicht im aquaristischen oder zoologischen Bereich \\ \hline
3. Smartphone-Kenntnisse und -Erfahrung	& Ein hoher Anteil in dieser Altersgruppe nutzt ein Smartphone und kennt sich dementsprechend gut aus \\ \hline
4. Fachwissen							& Der Benutzer ist ein Aquarium Neuling und hat dementsprechend noch kein oder wenig Fachwissen \\ \hline
5. Spezielle Produkterfahrung				& Möglicherweise hat der Benutzer bereits ein ähnliches System genutzt, welches Teilfunktionalitäten von unserer Anwendung besitzt \\ \hline
6. Motivation							& Benutzer in dieser Altersklasse und Erfahrungsstufe haben sich vermutlich vor kurzer Zeit ein Aquarium angeschafft oder überlegen noch, ob ein Aquarium angeschafft werden sollte. Da man natürlich das Beste aus seinem Aquarium rausholen möchte, bietet sich das System dem Benutzer gut an \\ \hline
7. Aufgaben							& - Fische füttern 							\\ 
									& - Wasserwechsel                       			\\
									& - Düngen                        				\\
									& - (Wasseranalyse durchführen)                    	\\ \hline
8. Auswirkung von Fehlern				& - Wasserverschmutzung 					\\
									& - Algenbildung                      				\\
									& - Sterben von Fischen und Pflanzen               \\ \hline
9. Verfügbare Technologien				& - Tröpfchen Tests 						\\ 
									& - (Technisches Gerät zur Wasseranalyse)		\\ \hline
\end{longtable}
\endgroup
\end{filecontents}
\LTXtable{\linewidth}{user_profile_2.auto}

\renewcommand{\title}{Fachhändler}
\section{\title}

Neben den normalen Benutzern stellt die Gruppe der Fachhändler eine wichtige Gruppe dar. Diese unterscheiden sich vor allem darin zu den normalen Benutzern, dass sie eine andere Anwendung benutzen und somit auch andere Aufgaben haben. Eine Unterteilung unter den Fachhändlern haben wir aber nicht für nötig gehalten, da die einzelnen Merkmale für jeden Fachhändler ungefähr gleich ausfallen sollten.

\begin{filecontents}{user_profile_8.auto}
\begingroup
\renewcommand{\arraystretch}{1.4} % Vertical Padding ändern
\begin{longtable}{%
|>{\raggedright\arraybackslash}X%
|>{\raggedright\arraybackslash}X%
|%
}
\caption{\title}	\\
\multicolumn{1}{c}{\textbf{Merkmal}}   	& \multicolumn{1}{c}{\textbf{Merkmalsausprägung}}		\\ \hline
\endfirsthead
\caption{\title - Fortsetzung}	\\
\multicolumn{1}{c}{\textbf{Merkmal}}   	& \multicolumn{1}{c}{\textbf{Merkmalsausprägung}}		\\ \hline	
\endhead
\hline
\endfoot
1. Demographisch 						& 								                         	\\
			          					& 								                       	\\
Alter			         					& 18 - 67										      \\
Geschlecht		    					& Männlich / Weiblich								\\
Wohnort                 					& Deutschland							                  	\\
Sozial-ökonomischer Status				& - Beruf im aquaristischen oder zoologischen Bereich		\\
									& - Variables Einkommen	                        			\\ \hline
2. Berufserfahrung						& Kurze bis lange Berufserfahrung im aquaristischen oder zoologischen Bereich \\ \hline
3. Computer-Kenntnisse und -Erfahrung	& Ein hoher Anteil in dieser Altersgruppe nutzt Computer und kennt sich dementsprechend gut aus \\ \hline
4. Fachwissen							& Der Benutzer arbeitet durch seinen Beruf im Themengebiet und kann somit als Fortgeschritten oder auch als Experte bezeichnet werden \\ \hline
5. Spezielle Produkterfahrung				& Möglicherweise hat der Benutzer bereits ein ähnliches System genutzt, welches Teilfunktionalitäten von unserer Anwendung besitzt \\ \hline
6. Motivation							& Fachhändler haben ein Interesse an der Zufriedenheit ihrer Kunden. Sie möchten ihnen helfen, ihre Aquarien optimal zu pflegen. Dabei helfen die Berechnungen der Wasserwerte sowie die Kommunikation, die über das System stattfinden \\ \hline
7. Aufgaben							& - Kunden beraten 						\\ 
									& - Probleme der Kunden lösen                       	\\
									& - Wasseranalyse durchführen                    	\\ \hline
8. Auswirkung von Fehlern				& - Wasserverschmutzung 					\\
									& - Algenbildung                      				\\
									& - Sterben von Fischen und Pflanzen               \\ \hline
9. Verfügbare Technologien				& - Technisches Gerät zur Wasseranalyse		\\ 
									& - (Tröpfchen Tests)						\\ \hline
\end{longtable}
\endgroup
\end{filecontents}
\LTXtable{\linewidth}{user_profile_8.auto}
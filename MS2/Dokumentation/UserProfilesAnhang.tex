\chapter{User Profiles}

\renewcommand{\title}{Kind/Jugendlicher}
\section{\title}

\begin{filecontents}{user_profile_1.auto}
\begingroup
\renewcommand{\arraystretch}{1.4} % Vertical Padding ändern
\begin{longtable}{%
|>{\raggedright\arraybackslash}X%
|>{\raggedright\arraybackslash}X%
|%
}
\caption{\title}	\\
\multicolumn{1}{c}{\textbf{Merkmal}}   	& \multicolumn{1}{c}{\textbf{Merkmalsausprägung}}		\\ \hline
\endfirsthead
\caption{\title - Fortsetzung}	\\
\multicolumn{1}{c}{\textbf{Merkmal}}   	& \multicolumn{1}{c}{\textbf{Merkmalsausprägung}}		\\ \hline	
\endhead
\hline
\endfoot
1. Demographisch 						& 								                         	\\
			          					& 								                       	\\
Alter			         					& 8 - 18										      \\
Geschlecht		    					& Männlich / Weiblich								\\
Wohnort                 					& Deutschland							                  	\\
Sozial-ökonomischer Status				& - Grundschule                        					\\
									& - Weiterführende Schule                        			\\
									& - Ausbildung                        						\\
									& - Wohnhaft bei den Eltern                        			\\
									& - i. d. R. kein Einkommen (außer bei Ausbildung)		\\ \hline
2. Berufserfahrung						& Mögliche (geringe) Berufserfahrung durch Ferienjobs (ab 16 Jahren) oder Ausbildung (auch ca. ab 16 Jahren), ansonsten i. d. R. keine Berufserfahrung \\ \hline
3. Smartphone-Kenntnisse und -Erfahrung	& Ein hoher Anteil in dieser Altersgruppe nutzt ein Smartphone und kennt sich dementsprechend gut aus \\ \hline
4. Fachwissen							& Benutzer in dieser Altersklasse haben in der Regel wenig Fachwissen über Aquaristik \\ \hline
5. Spezielle Produkterfahrung				& Möglicherweise hat der Benutzer bereits ein ähnliches System genutzt, welches Teilfunktionalitäten von unserer Anwendung besitzt \\ \hline
6. Motivation							& Benutzer in dieser Altersklasse sind vermutlich diejenigen, die sich ein Aquarium im Haushalt am meisten gewünscht haben. Allerdings haben sie in der Regel nicht so viel Verantwortung, außer wahrscheinlich das Füttern der Fische oder Ähnliches. Die Motivation zur Benutzung unseres Systems hängt dann davon ab, inwiefern sich der Benutzer auch noch um weitere Tätigkeiten rund um das Aquarium kümmern möchte, zum Beispiel, wenn es um die Wasserqualität geht \\ \hline
7. Aufgaben							& - Fische füttern \\ 
									& - (Wasserwechsel)                        			\\
									& - (Düngen)                        					\\ \hline
8. Auswirkung von Fehlern				& - Wasserverschmutzung \\
									& - Algenbildung                      			\\
									& - Sterben von Fischen und Pflanzen                        					\\ \hline
9. Verfügbare Technologien				& - Tröpfchen Tests \\ \hline
\end{longtable}
\endgroup
\end{filecontents}
\LTXtable{\linewidth}{user_profile_1.auto}

\renewcommand{\title}{Erwachsener - Aquarium Neuling}
\section{\title}

\begin{filecontents}{user_profile_2.auto}
\begingroup
\renewcommand{\arraystretch}{1.4} % Vertical Padding ändern
\begin{longtable}{%
|>{\raggedright\arraybackslash}X%
|>{\raggedright\arraybackslash}X%
|%
}
\caption{\title}	\\
\multicolumn{1}{c}{\textbf{Merkmal}}   	& \multicolumn{1}{c}{\textbf{Merkmalsausprägung}}		\\ \hline
\endfirsthead
\caption{\title - Fortsetzung}	\\
\multicolumn{1}{c}{\textbf{Merkmal}}   	& \multicolumn{1}{c}{\textbf{Merkmalsausprägung}}		\\ \hline	
\endhead
\hline
\endfoot
1. Demographisch 						& 								                         	\\
			          					& 								                       	\\
Alter			         					& 18 - 67										      \\
Geschlecht		    					& Männlich / Weiblich								\\
Wohnort                 					& Deutschland							                  	\\
Sozial-ökonomischer Status				& - Kein Beruf / Ausbildung / Studium im aquaristischen oder zoologischen Bereich   \\
									& - Variables Einkommen	                        			\\ \hline
2. Berufserfahrung						& Kurze bis lange Berufserfahrung, allerdings nicht im aquaristischen oder zoologischen Bereich \\ \hline
3. Smartphone-Kenntnisse und -Erfahrung	& Ein hoher Anteil in dieser Altersgruppe nutzt ein Smartphone und kennt sich dementsprechend gut aus \\ \hline
4. Fachwissen							& Der Benutzer ist ein Aquarium Neuling und hat dementsprechend noch kein oder wenig Fachwissen \\ \hline
5. Spezielle Produkterfahrung				& Möglicherweise hat der Benutzer bereits ein ähnliches System genutzt, welches Teilfunktionalitäten von unserer Anwendung besitzt \\ \hline
6. Motivation							& Benutzer in dieser Altersklasse und Erfahrungsstufe haben sich vermutlich vor kurzer Zeit ein Aquarium angeschafft oder überlegen noch, ob ein Aquarium angeschafft werden sollte. Da man natürlich das Beste aus seinem Aquarium rausholen möchte, bietet sich das System dem Benutzer gut an \\ \hline
7. Aufgaben							& - Fische füttern 							\\ 
									& - Wasserwechsel                       			\\
									& - Düngen                        				\\
									& - (Wasseranalyse durchführen)                    	\\ \hline
8. Auswirkung von Fehlern				& - Wasserverschmutzung 					\\
									& - Algenbildung                      				\\
									& - Sterben von Fischen und Pflanzen               \\ \hline
9. Verfügbare Technologien				& - Tröpfchen Tests 						\\ 
									& - (Technisches Gerät zur Wasseranalyse)		\\ \hline
\end{longtable}
\endgroup
\end{filecontents}
\LTXtable{\linewidth}{user_profile_2.auto}

\renewcommand{\title}{Erwachsener - Aquarium Fortgeschritten}
\section{\title}

\begin{filecontents}{user_profile_3.auto}
\begingroup
\renewcommand{\arraystretch}{1.4} % Vertical Padding ändern
\begin{longtable}{%
|>{\raggedright\arraybackslash}X%
|>{\raggedright\arraybackslash}X%
|%
}
\caption{\title}	\\
\multicolumn{1}{c}{\textbf{Merkmal}}   	& \multicolumn{1}{c}{\textbf{Merkmalsausprägung}}		\\ \hline
\endfirsthead
\caption{\title - Fortsetzung}	\\
\multicolumn{1}{c}{\textbf{Merkmal}}   	& \multicolumn{1}{c}{\textbf{Merkmalsausprägung}}		\\ \hline	
\endhead
\hline
\endfoot
1. Demographisch 						& 								                         	\\
			          					& 								                       	\\
Alter			         					& 18 - 67										      \\
Geschlecht		    					& Männlich / Weiblich								\\
Wohnort                 					& Deutschland							                  	\\
Sozial-ökonomischer Status				& - Möglicherweise Beruf / Ausbildung / Studium im aquaristischen oder zoologischen Bereich   \\
									& - Variables Einkommen	                        			\\ \hline
2. Berufserfahrung						& Kurze bis lange Berufserfahrung, möglicherweise im aquaristischen oder zoologischen Bereich \\ \hline
3. Smartphone-Kenntnisse und -Erfahrung	& Ein hoher Anteil in dieser Altersgruppe nutzt ein Smartphone und kennt sich dementsprechend gut aus \\ \hline
4. Fachwissen							& Der Benutzer ist möglicherweise bereits durch seinen Beruf / Ausbildung / Studium oder auch durch sein Hobby fortgeschritten, was die Erfahrung mit Aquarien angeht. \\ \hline
5. Spezielle Produkterfahrung				& Möglicherweise hat der Benutzer bereits ein ähnliches System genutzt, welches Teilfunktionalitäten von unserer Anwendung besitzt \\ \hline
6. Motivation							& Benutzer in dieser Altersklasse und Erfahrungsstufe haben vermutlich schon etwas länger ein Aquarium und wollen nun ihre Abläufe optimieren. Dabei ist unser System eine hilfreiche Anwendung \\ \hline
7. Aufgaben							& - Fische füttern 							\\ 
									& - Wasserwechsel                       			\\
									& - Düngen                        				\\
									& - (Wasseranalyse durchführen)                    	\\ \hline
8. Auswirkung von Fehlern				& - Wasserverschmutzung 					\\
									& - Algenbildung                      				\\
									& - Sterben von Fischen und Pflanzen               \\ \hline
9. Verfügbare Technologien				& - Tröpfchen Tests 						\\ 
									& - (Technisches Gerät zur Wasseranalyse)		\\ \hline
\end{longtable}
\endgroup
\end{filecontents}
\LTXtable{\linewidth}{user_profile_3.auto}

\renewcommand{\title}{Erwachsener - Aquarium Experte}
\section{\title}

\begin{filecontents}{user_profile_4.auto}
\begingroup
\renewcommand{\arraystretch}{1.4} % Vertical Padding ändern
\begin{longtable}{%
|>{\raggedright\arraybackslash}X%
|>{\raggedright\arraybackslash}X%
|%
}
\caption{\title}	\\
\multicolumn{1}{c}{\textbf{Merkmal}}   	& \multicolumn{1}{c}{\textbf{Merkmalsausprägung}}		\\ \hline
\endfirsthead
\caption{\title - Fortsetzung}	\\
\multicolumn{1}{c}{\textbf{Merkmal}}   	& \multicolumn{1}{c}{\textbf{Merkmalsausprägung}}		\\ \hline	
\endhead
\hline
\endfoot
1. Demographisch 						& 								                         	\\
			          					& 								                       	\\
Alter			         					& 18 - 67										      \\
Geschlecht		    					& Männlich / Weiblich								\\
Wohnort                 					& Deutschland							                  	\\
Sozial-ökonomischer Status				& - Vermutlich Beruf im aquaristischen oder zoologischen Bereich   \\
									& - Variables Einkommen	                        			\\ \hline
2. Berufserfahrung						& Vermutlich Berufserfahrung im aquaristischen oder zoologischen Bereich \\ \hline
3. Smartphone-Kenntnisse und -Erfahrung	& Ein hoher Anteil in dieser Altersgruppe nutzt ein Smartphone und kennt sich dementsprechend gut aus \\ \hline
4. Fachwissen							& Der Benutzer arbeitet durch seinen Beruf intensiv im Themengebiet oder hat sich in seiner Freizeit intensiv mit dem Thema beschäftigt und kann somit als Experte bezeichnet werden \\ \hline
5. Spezielle Produkterfahrung				& Möglicherweise hat der Benutzer bereits ein ähnliches System genutzt, welches Teilfunktionalitäten von unserer Anwendung besitzt \\ \hline
6. Motivation							& Benutzer in dieser Altersklasse und Erfahrungsstufe haben neben ihrem Beruf möglicherweise auch ein privates Interesse an Aquarien. Um ihre Abläufe zu optimieren, bietet sich unser System sehr gut an \\ \hline
7. Aufgaben							& - Fische füttern 							\\ 
									& - Wasserwechsel                       			\\
									& - Düngen                        				\\
									& - (Wasseranalyse durchführen)                    	\\ \hline
8. Auswirkung von Fehlern				& - Wasserverschmutzung 					\\
									& - Algenbildung                      				\\
									& - Sterben von Fischen und Pflanzen               \\ \hline
9. Verfügbare Technologien				& - Tröpfchen Tests 						\\ 
									& - (Technisches Gerät zur Wasseranalyse)		\\ \hline
\end{longtable}
\endgroup
\end{filecontents}
\LTXtable{\linewidth}{user_profile_4.auto}

\renewcommand{\title}{Rentner - Aquarium Neuling}
\section{\title}

\begin{filecontents}{user_profile_5.auto}
\begingroup
\renewcommand{\arraystretch}{1.4} % Vertical Padding ändern
\begin{longtable}{%
|>{\raggedright\arraybackslash}X%
|>{\raggedright\arraybackslash}X%
|%
}
\caption{\title}	\\
\multicolumn{1}{c}{\textbf{Merkmal}}   	& \multicolumn{1}{c}{\textbf{Merkmalsausprägung}}		\\ \hline
\endfirsthead
\caption{\title - Fortsetzung}	\\
\multicolumn{1}{c}{\textbf{Merkmal}}   	& \multicolumn{1}{c}{\textbf{Merkmalsausprägung}}		\\ \hline	
\endhead
\hline
\endfoot
1. Demographisch 						& 								                         	\\
			          					& 								                       	\\
Alter			         					& 67+										      \\
Geschlecht		    					& Männlich / Weiblich								\\
Wohnort                 					& Deutschland							                  	\\
Sozial-ökonomischer Status				& - Rentner										\\
									& - Renten Einkommen	                        			\\ \hline
2. Berufserfahrung						& Sehr lange Berufserfahrung, aber nicht im aquaristischen oder zoologischen Bereich \\ \hline
3. Smartphone-Kenntnisse und -Erfahrung	& Ein eher geringer Anteil in dieser Altersgruppe benutzt Smartphones \\ \hline
4. Fachwissen							& Der Benutzer ist ein Aquarium Neuling und hat dementsprechend noch kein oder wenig Fachwissen \\ \hline
5. Spezielle Produkterfahrung				& Möglicherweise hat der Benutzer bereits ein ähnliches System genutzt, welches Teilfunktionalitäten von unserer Anwendung besitzt \\ \hline
6. Motivation							& Benutzer in dieser Altersklasse und Erfahrungsstufe haben sich vermutlich vor kurzer Zeit ein Aquarium angeschafft oder überlegen noch, ob ein Aquarium angeschafft werden sollte. Da sie (in der Regel) keinen Beruf mehr ausüben, haben sie sehr viel Zeit und da bietet sich ein Aquarium gut an und da man natürlich das Beste aus seinem Aquarium rausholen möchte, besteht das Interesse an der Nutzung unseres Systems \\ \hline
7. Einschränkungen						& Aufgrund des Alters haben die Benutzer möglicherweise Einschränkungen was das Sehen betrifft oder andere körperliche Einschränkungen \\ \hline
8. Aufgaben							& - Fische füttern 							\\ 
									& - Wasserwechsel                       			\\
									& - Düngen                        				\\
									& - (Wasseranalyse durchführen)                    	\\ \hline
9. Auswirkung von Fehlern				& - Wasserverschmutzung 					\\
									& - Algenbildung                      				\\
									& - Sterben von Fischen und Pflanzen               \\ \hline
10. Verfügbare Technologien				& - Tröpfchen Tests 						\\ 
									& - (Technisches Gerät zur Wasseranalyse)		\\ \hline
\end{longtable}
\endgroup
\end{filecontents}
\LTXtable{\linewidth}{user_profile_5.auto}

\renewcommand{\title}{Rentner - Aquarium Fortgeschritten}
\section{\title}

\begin{filecontents}{user_profile_6.auto}
\begingroup
\renewcommand{\arraystretch}{1.4} % Vertical Padding ändern
\begin{longtable}{%
|>{\raggedright\arraybackslash}X%
|>{\raggedright\arraybackslash}X%
|%
}
\caption{\title}	\\
\multicolumn{1}{c}{\textbf{Merkmal}}   	& \multicolumn{1}{c}{\textbf{Merkmalsausprägung}}		\\ \hline
\endfirsthead
\caption{\title - Fortsetzung}	\\
\multicolumn{1}{c}{\textbf{Merkmal}}   	& \multicolumn{1}{c}{\textbf{Merkmalsausprägung}}		\\ \hline	
\endhead
\hline
\endfoot
1. Demographisch 						& 								                         	\\
			          					& 								                       	\\
Alter			         					& 67+										      \\
Geschlecht		    					& Männlich / Weiblich								\\
Wohnort                 					& Deutschland							                  	\\
Sozial-ökonomischer Status				& - Rentner										\\
									& - Renten Einkommen	                        			\\ \hline
2. Berufserfahrung						& Sehr lange Berufserfahrung; möglicherweise im aquaristischen oder zoologischen Bereich \\ \hline
3. Smartphone-Kenntnisse und -Erfahrung	& Ein eher geringer Anteil in dieser Altersgruppe benutzt Smartphones \\ \hline
4. Fachwissen							& Der Benutzer ist möglicherweise bereits durch seinen ehemaligen Beruf / Ausbildung / Studium oder auch durch sein Hobby fortgeschritten, was die Erfahrung mit Aquarien angeht \\ \hline
5. Spezielle Produkterfahrung				& Möglicherweise hat der Benutzer bereits ein ähnliches System genutzt, welches Teilfunktionalitäten von unserer Anwendung besitzt \\ \hline
6. Motivation							& Benutzer in dieser Altersklasse und Erfahrungsstufe haben vermutlich schon etwas länger ein Aquarium und wollen nun ihre Abläufe optimieren. Da sie (in der Regel) keinen Beruf mehr ausüben, haben sie sehr viel Zeit und da bietet sich ein Aquarium gut an und da man natürlich das Beste aus seinem Aquarium rausholen möchte, besteht das Interesse an der Nutzung unseres Systems \\ \hline
7. Einschränkungen						& Aufgrund des Alters haben die Benutzer möglicherweise Einschränkungen was das Sehen betrifft oder andere körperliche Einschränkungen \\ \hline
8. Aufgaben							& - Fische füttern 							\\ 
									& - Wasserwechsel                       			\\
									& - Düngen                        				\\
									& - (Wasseranalyse durchführen)                    	\\ \hline
9. Auswirkung von Fehlern				& - Wasserverschmutzung 					\\
									& - Algenbildung                      				\\
									& - Sterben von Fischen und Pflanzen               \\ \hline
10. Verfügbare Technologien				& - Tröpfchen Tests 						\\ 
									& - (Technisches Gerät zur Wasseranalyse)		\\ \hline
\end{longtable}
\endgroup
\end{filecontents}
\LTXtable{\linewidth}{user_profile_6.auto}

\renewcommand{\title}{Rentner - Aquarium Experte}
\section{\title}

\begin{filecontents}{user_profile_7.auto}
\begingroup
\renewcommand{\arraystretch}{1.4} % Vertical Padding ändern
\begin{longtable}{%
|>{\raggedright\arraybackslash}X%
|>{\raggedright\arraybackslash}X%
|%
}
\caption{\title}	\\
\multicolumn{1}{c}{\textbf{Merkmal}}   	& \multicolumn{1}{c}{\textbf{Merkmalsausprägung}}		\\ \hline
\endfirsthead
\caption{\title - Fortsetzung}	\\
\multicolumn{1}{c}{\textbf{Merkmal}}   	& \multicolumn{1}{c}{\textbf{Merkmalsausprägung}}		\\ \hline	
\endhead
\hline
\endfoot
1. Demographisch 						& 								                         	\\
			          					& 								                       	\\
Alter			         					& 67+										      \\
Geschlecht		    					& Männlich / Weiblich								\\
Wohnort                 					& Deutschland							                  	\\
Sozial-ökonomischer Status				& - Rentner										\\
									& - Renten Einkommen	                        			\\ \hline
2. Berufserfahrung						& Vermutlich lange Berufserfahrung im aquaristischen oder zoologischen Bereich \\ \hline
3. Smartphone-Kenntnisse und -Erfahrung	& Ein eher geringer Anteil in dieser Altersgruppe benutzt Smartphones \\ \hline
4. Fachwissen							& Der Benutzer hat in seinem ehemaligen Beruf intensiv im Themengebiet gearbeitet oder hat sich in seiner Freizeit intensiv mit dem Thema beschäftigt und kann somit als Experte bezeichnet werden \\ \hline
5. Spezielle Produkterfahrung				& Möglicherweise hat der Benutzer bereits ein ähnliches System genutzt, welches Teilfunktionalitäten von unserer Anwendung besitzt \\ \hline
6. Motivation							& Benutzer in dieser Altersklasse und Erfahrungsstufe haben vermutlich schon länger ein Aquarium und wollen nun ihre Abläufe optimieren. Da sie (in der Regel) keinen Beruf mehr ausüben, haben sie sehr viel Zeit und da bietet sich ein Aquarium gut an und da man natürlich das Beste aus seinem Aquarium rausholen möchte, besteht das Interesse an der Nutzung unseres Systems \\ \hline
7. Einschränkungen						& Aufgrund des Alters haben die Benutzer möglicherweise Einschränkungen was das Sehen betrifft oder andere körperliche Einschränkungen \\ \hline
8. Aufgaben							& - Fische füttern 							\\ 
									& - Wasserwechsel                       			\\
									& - Düngen                        				\\
									& - (Wasseranalyse durchführen)                    	\\ \hline
9. Auswirkung von Fehlern				& - Wasserverschmutzung 					\\
									& - Algenbildung                      				\\
									& - Sterben von Fischen und Pflanzen               \\ \hline
10. Verfügbare Technologien				& - Tröpfchen Tests 						\\ 
									& - (Technisches Gerät zur Wasseranalyse)		\\ \hline
\end{longtable}
\endgroup
\end{filecontents}
\LTXtable{\linewidth}{user_profile_7.auto}

\renewcommand{\title}{Fachhändler}
\section{\title}

\begin{filecontents}{user_profile_8.auto}
\begingroup
\renewcommand{\arraystretch}{1.4} % Vertical Padding ändern
\begin{longtable}{%
|>{\raggedright\arraybackslash}X%
|>{\raggedright\arraybackslash}X%
|%
}
\caption{\title}	\\
\multicolumn{1}{c}{\textbf{Merkmal}}   	& \multicolumn{1}{c}{\textbf{Merkmalsausprägung}}		\\ \hline
\endfirsthead
\caption{\title - Fortsetzung}	\\
\multicolumn{1}{c}{\textbf{Merkmal}}   	& \multicolumn{1}{c}{\textbf{Merkmalsausprägung}}		\\ \hline	
\endhead
\hline
\endfoot
1. Demographisch 						& 								                         	\\
			          					& 								                       	\\
Alter			         					& 18 - 67										      \\
Geschlecht		    					& Männlich / Weiblich								\\
Wohnort                 					& Deutschland							                  	\\
Sozial-ökonomischer Status				& - Beruf im aquaristischen oder zoologischen Bereich		\\
									& - Variables Einkommen	                        			\\ \hline
2. Berufserfahrung						& Kurze bis lange Berufserfahrung im aquaristischen oder zoologischen Bereich \\ \hline
3. Computer-Kenntnisse und -Erfahrung	& Ein hoher Anteil in dieser Altersgruppe nutzt Computer und kennt sich dementsprechend gut aus \\ \hline
4. Fachwissen							& Der Benutzer arbeitet durch seinen Beruf im Themengebiet und kann somit als Fortgeschritten oder auch als Experte bezeichnet werden \\ \hline
5. Spezielle Produkterfahrung				& Möglicherweise hat der Benutzer bereits ein ähnliches System genutzt, welches Teilfunktionalitäten von unserer Anwendung besitzt \\ \hline
6. Motivation							& Fachhändler haben ein Interesse an der Zufriedenheit ihrer Kunden. Sie möchten ihnen helfen, ihre Aquarien optimal zu pflegen. Dabei helfen die Berechnungen der Wasserwerte sowie die Kommunikation, die über das System stattfinden \\ \hline
7. Aufgaben							& - Kunden beraten 						\\ 
									& - Probleme der Kunden lösen                       	\\
									& - Wasseranalyse durchführen                    	\\ \hline
8. Auswirkung von Fehlern				& - Wasserverschmutzung 					\\
									& - Algenbildung                      				\\
									& - Sterben von Fischen und Pflanzen               \\ \hline
9. Verfügbare Technologien				& - Technisches Gerät zur Wasseranalyse		\\ 
									& - (Tröpfchen Tests)						\\ \hline
\end{longtable}
\endgroup
\end{filecontents}
\LTXtable{\linewidth}{user_profile_8.auto}
\chapter{Wahl des Vorgehensmodells}
Von Beginn an war uns deutlich, dass die Wahl auf einen benutzer-zentrierten Ansatz fallen muss. Da mit der speziellen Domäne der Aquarianer und den zugehörigen Fachhandeln auf eine Zielgruppe eingegangen werden soll, welche einen großen Unterschied in der Bandbreite an bereits vorhandenen Fähigkeiten besitzt. Weil der benutzungs-zentrierte Ansatz dies aber weniger beachtet, war das bereits ein Auschlusskriterium.\\
 
Nach längerer Überlegung haben wir uns schließlich für den Usability Engineering Lifecycle von Deborah Mayhew entschieden. Mayhew geht in ihrem Buch `"The Usability Engineering Lifecycle" detailliert auf ihr Vorgehen ein und bietet eine Menge an Beispielen, möglichen Abkürzungen, dem geschätzen Aufwand, detaillierten Beschreibungen zum Vorgehen fürs Anfertigen der Artefakte und den Zirkelbezügen zwischen den Artefakten. Für einen Einstieg in das Usability Engineering ist dieser Vorgang also sehr gut geeignet. 
Die Verwendung der Shortcuts wird in \ref{ue_cycle} durch gestrichelte Pfeile dargestellt, diese Abkürzungen sind aufgrund des geschätzten Aufwandes von etwa 2400 Stunden unumgänglich.\\
Mayhew erwähnt für einfache Projekte, mit minimalen Ressourcen zu Beginn des Buches eine mögliche Reduzierung des Lifecycles\cite[25]{Mayhew:UEL}, darunter fallen das Anfertigen von 
\begin{itemize}
 \item "Quick and dirty" - Versionen der User Profiles und der Aufgabenanalyse,
 \item der Reduzierung auf Design Prinzipen der bereits vorhandenen Literatur,
 \item das Zusammenfügen der drei Design-Levels in einen einfach iterativen Designprozess und
 \item dem Überspringen der Anfertigung eines Style-Guides für das entstehende Produkt.
\end{itemize}

Unter anderem war uns noch die Übertragbarkeit von den schon "älteren" Erkenntnissen des Usability Engineerings auf das Entwickeln aktueller interaktive Systeme wichtig, hier liefert Mayhew aber bereits einen Puffer durch ihre Aussage "Although the lifecycle is described mostly in the context of developing typical office software applications, it is equally applicable to projects developing any kind of interactive product".\cite[5]{Mayhew:UEL}

Durch die einzelnen Shortcuts besteht auch noch die Möglichkeit, dass Benutzer des Systems erst im letzten Schritt hinzugezogen werden, dies ist statt eines Nachteils für uns ein besonderer Vorteil, da Johannes bereits Erfahrungen im Bereich der Aquaristik besitzt und so mögliche Zeitersparnisse resultieren können. Das benutzer-zentrierte Gestalten geht zwar von fehlender Perspektive und Aufgaben der Entwickler in der entsprechenden Domäne aus, was aber aufgrund von fehlender Erfahrung beim zweiten Teammitglied und den unterschiedlichen Erfahrungen der zukünftigen Benutzer des Systems von geringer Relevanz ist. 

Die Vorteile der detaillierten Beschreibung, der hohen Skalierbarkeit, der Anpassung an moderne Projekte und zuletzt auch noch die Tatsache, dass die Benutzer des Systems erst in der letzten Phase hinzugezogen werden, führten dazu, dass wir uns letztendlich für den Usability Engineering Lifecycle von Deborah Mayhew entschieden haben.

\includegraphics{ue_cycle}\label{ue_cycle}
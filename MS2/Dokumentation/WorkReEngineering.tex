\chapter{Work Re-Engineering}
Das Ziel des Work Re-Engineering ist nicht die Nachahmung des aktuellen Modells oder die Revolutionierung der Aufgabenbearbeitung, sondern viel mehr die Kombination zwischen der möglichen Effizienzsteigerung durch die Automatisierung der Prozesse, dem effizienteren Erreichen der Ziele und einem einfacheren Wiedereinstieg in das neue System unter Betrachtung der Einflüsse der Mensch-Computer-Interaktion.

Möglichkeiten um die Arbeit zu reduzieren durchs Task Model oder Arbeitsprozesse rationalisieren(optimale Zieleerreichung durch neue Bedingungen)
Changing work practice ist gerechtfertig, wenn nicht nur automatisierung sondern auch Usability goals erreicht werden.
Ein wenig Änderung ist benötigt

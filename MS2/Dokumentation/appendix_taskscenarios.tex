\section{Anhang: TaskSzenarios}\label{app:taskscenarios}

\subsection{Task: Dokumentation der Wasserwerte}
\textbf{User:} Aquarienbesitzer\\

\textbf{Definition:} Der Aquarienbesitzer ist zu Hause und interessiert sich für die Wasserwerte in seinem Aquarium, da er sich für eine bestimmte neue Fischart interessiert. Um diese Fischart halten zu können, sind bestimmte Nährwerte im Wasser benötigt, damit diese Fische überleben.\\

\textbf{Task Flow:}
\begin{enumerate}
 	\item Toni, der Aquarienbesitzer, geht zum Fachhandel mit seiner Wasserprobe und lässt diese analysieren.
 	\item Der Fachhandel analysiert diese Wasserwerte, während Toni wieder nach Hause fährt und auf die Ergebnisse wartet.
 	\item Nach ein paar Tagen erhält Toni eine Antwort vom Fachhandel, wie denn seine Wasserwerte momentan aussehen. Dazu fährt Toni wieder zum Fachhandel und lässt sich  hier auch noch die tägliche Änderungsrate dank einer alten Analyse berechnen.
 	\item Toni kennt nun seine Wasserwerte und weiß, dass seine Wasserwerte für den neuen Fisch optimal sind und legt sich daraufhin diesen neuen Fisch zu.
	 \item Der neue Fisch lebt ohne Probleme im neuen Aquarium und passt zum Gesamtbild sehr gut dazu.
\end{enumerate}
\textbf{Task Closure:} Dieses Szenarios nahm 3 Tage in Anspruch, am ersten Tag bringt Toni die Wasserwerte zur Fachhandlung, am 2. Tag findet die Analyse statt und am 3. Tag holt Toni sich die Wasserwerte ab und kauft sich den neuen Fisch.\\

Um diese Aufgabe zu unterstützen, sollte das User Interface...
\begin{itemize}
  \item Die tägliche Änderungsrate der Nährwerte automatisch berechnen
  \item Die Dokumentation der Wasserwerte zu digitalisieren
\end{itemize}	


\subsection{Task: Wasserwechsel durchführen}

\textbf{User:} Aquarienbesitzer\\

\textbf{Description:} Zur guten Aquarienpflege gehört ein wöchentlicher Wasserwechsel. Da das Grundwasser an manchen Wohnorten abweichende Werte zu den Zielwerten des Aquariumwassers besitzt, ist eine gezielte Dosierung von Osmose- und Grundwasser selten zu vermeiden.\\

\textbf{Task Flow:}
\begin{enumerate}
 \item Mia wohnt in der Nähe eines Kanals, wo das Grundwasser aus dem Kanal abgepumpt wird, dadurch weichen die Wasserwerte des Grundwassers stark von den optimalen Werten für ihr Aquarium ab. Da ihr die Pflege ihres Aquariums aber trotzdem sehr wichtig ist, führt sie einen Wasserwechsel mit Berücksichtigung der Zielwerte durch
 \item Zunächst analysiert Mia die Werte ihres aktuellen Aquariumwassers und erhält Gesamt- und Karbonathärte sowie den pH-Wert.
 \item Mit Hilfe der Grundwasserwerte vom Versorger kann sie anschließend das benötigte Verhältnis des Osmose- und des Grundwasser unter Berücksichtigung der Menge des Wasserwechsels per Hand berechnen.
 \item Mia mischt das Grund- und Osmosewasser und führt den Wasserwechsel in ihrem Aquarium durch.
 \item Da Mia die Werte ihres Aquariums sorgfältig dokumentiert, ist das Eintragen des neuen Wasserwertes sehr einfach.
\end{enumerate}

\textbf{Task Closure:} Die Durchführung der Aufgabe dauert nur 40 Minuten. 10 Minuten für die Berechnung, 10 Minuten für die Analyse und noch 15 Minuten für die Durchführung der Wasserprobe und 5 weitere Minuten für das Eintragen der Dokumentation\\

Um diese Aufgabe zu unterstützen, sollte das User Interface...
\begin{itemize}
  \item Die Berechnung des Aquarienbesitzer durchführen
  \item An die Durchführung des Wasserwechsels erinnern
  \item Die Dokumentation der Wasserwerte übernehmen und die Durchführung eines Wasserwechsels bereits vor der Durchführung anzuzeigen
\end{itemize}
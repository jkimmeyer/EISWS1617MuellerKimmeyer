\documentclass[a4paper,ngerman]{scrreprt}
\usepackage[T1]{fontenc}
\usepackage[utf8]{inputenc}
\usepackage{babel}
\usepackage{array}
\usepackage{tabularx}
\usepackage[babel, german=quotes]{csquotes}
\usepackage{biblatex}
\bibliography{literatur}

\begin{document}

\section{User Profiles}

\begin{table}[htb]
\centering
\caption{Plattform: Android App}
\begingroup
\renewcommand{\arraystretch}{1.4} % Vertical Padding ändern
\begin{tabularx}{\linewidth}{%
|>{\raggedright\arraybackslash}X%
|>{\raggedright\arraybackslash}X%
|%
}
\hline
\textbf{Merkmal}   	& \textbf{Merkmalsausprägung}                        			    		\\ \hline
Betriebssystem Version 	& 4.0.3 und höher                         					\\ \hline
Display Größe          	& 4 Zoll und größer                        			            	\\ \hline
Eingabe-Geräte         	& Virtuelle Tastatur, Touchscreen         				\\ \hline
Internetverbindung     	& Verbindung über WLAN, mobile Verbindung 			\\ \hline
Farben                 	& Alle beliebigen Farben                  					\\ \hline
Spezial-Effekte        	& 3D, Video, Audio                        					\\ \hline
GUI Werkzeuge          	& Siehe Android Komponenten \autocite{Android:Komponenten}  \\ \hline
Energieversorgung      	& Begrenzte Akkulaufzeit, Netzbetrieb     				\\ \hline
Multitasking           	& x                                       						\\ \hline
Bit-Mapped-Display     	& x                                       						\\ \hline
Windowing              	&                                         						\\ \hline
\end{tabularx}
\endgroup
\end{table}

\printbibliography

\end{document}
\chapter{Design Änderungen}

Bei der App habe ich doch eine andere Struktur, als in den Detailed User Interfaces geplant war, umgesetzt. In der Projekt Dokumentation hatten wir uns entschieden, eine Tab-Struktur zu nehmen. Nach erneutem Überlegen habe ich mich aber dagegen entschieden und eine Sidemenu-Struktur genommen, da eine Tab-Strukur eher typisch für iOS gewesen wäre und nicht für Android. Das lässt sich u.a. damit begründen, dass Android Geräte meistens unten am Bildschirm zusätzliche ``Buttons'' haben, wie zum Beispiel den ``Zurück-Button''. Wenn die Tabs wie geplant am unteren Bildschirmrand platziert worden wären, hätte es leicht passieren können, dass man versehentlich auf einen Tab geklickt hätte, obwohl man den ``Zurück-Button'' antippen wollte oder umgekehrt. Da wir aber beim konzeptuellen Modell sowieso eine Sidemenu-Struktur als Alternative in Betracht gezogen hatten, konnte ich mich daran orientieren.
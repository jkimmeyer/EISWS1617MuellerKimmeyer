\chapter{Installationsdokumentation}

Bevor ich genauer auf die Implementierung eingehen werde, erkläre ich kurz den Installationsvorgang, um die Anwendungen zum Laufen zu bekommen. Die Ordner befinden sich jeweils in unserem Repository (\url{https://github.com/jkimmeyer/EISWS1617MuellerKimmeyer}) unter ``MS3/Implementation/''.

\section{Server}

Den Server habe ich auf uberspace.de gehostet, sodass er durchgehen läuft und von jedem Gerät aus erreichbar ist, ohne dass erst immer der Server gestartet werden muss. Der Server ist unter (\url{http://eis1617.lupus.uberspace.de/nodejs}) zu erreichen. Als Ressource kann zum Beispiel ``users'' angegeben werden. Um den Server zu testen können Requests zum Beispiel von der Seite \url{https://www.hurl.it/} abgeschickt werden. Dabei sollte man beachten, dass der Header ``Content-Type'' gleich ``application/json'' gesetzt werden sollte. Im Body sollte ein JSON String mit den entsprechenden Parametern stehen. Wenn Sie den Server auch lokal testen wollen, müssen Sie folgende Installationsschritte vornehmen:

\subsubsection{Vorbereitungen}

\begin{enumerate}
\item MongoDB installieren (\url{https://www.mongodb.com/download-center#community})
\item Folgenden Ordner manuell erstellen: C:{\textbackslash}data{\textbackslash}db
\item Die Datei C:{\textbackslash}Program Files{\textbackslash}MongoDB{\textbackslash}Server{\textbackslash}3.4{\textbackslash}bin{\textbackslash}mongod.exe ausführen
\item Der Server läuft nun
\end{enumerate}

\subsubsection{Server Installation}

\begin{enumerate}
\item Repository (\url{https://github.com/jkimmeyer/EISWS1617MuellerKimmeyer}) runterladen und in den Ordner /MS3/Implementation/server gehen
\item Die Konsole in diesem Ordner öffnen und folgenden Befehl eingeben: npm install
\item Die Datei server.js öffnen und Zeile 23 auskommentieren und Zeile 25 entfernen
\item Nun in der Konsole eingeben: node server.js
\item Jetzt können Sie im Browser die Adresse localhost:61000 aufrufen oder mit dem REST Client Ihrer Wahl Requests an den Server schicken
\end{enumerate}

\section{Benutzer Client - Android App}

\subsubsection{Vorraussetzungen}

\begin{enumerate}
\item Android Smartphone oder Android Studio Entwicklungsumgebung mit einem Virtual Device
\end{enumerate}

\subsubsection{Vorbereitung}

\begin{enumerate}
\item Android Studio installieren (\url{https://developer.android.com/studio/index.html})
\end{enumerate}

\subsubsection{Projekt importieren}

\begin{enumerate}
\item Am besten den Ordner ``App'' unter MS3/Implementation/benutzerClient auf den Desktop verschieben, da der Dateipfad sonst für manche Dateien zu lang ist und die nicht mehr von Android Studio erkannt werden.
\item In Android Studio unter File -> Open... den Dateipfad angeben und das Projekt importieren
\end{enumerate}

\subsubsection{App starten}

\begin{enumerate}
\item Oben in der Menüleiste auf den grünen Pfeil klicken
\item Dann ein virtuelles Device erstellen oder ein Android Smartphone anschließen und dieses auswählen
\end{enumerate}

\subsubsection{APK Datei auf Smartphone installieren}

Im Ordner ``Implementation'' befindet sich eine APK Datei, welche auf einem Android Smartphone installiert werden kann.

\section{Fachhändler Client - Desktop Anwendung}

\subsubsection{Vorraussetzungen}

\begin{enumerate}
\item Windows Betriebssystem
\end{enumerate}

\subsubsection{Vorbereitung}

\begin{enumerate}
\item Eclipse installieren (\url{https://www.eclipse.org/downloads/})
\end{enumerate}

\subsubsection{Projekt importieren}

\begin{enumerate}
\item Repository (\url{https://github.com/jkimmeyer/EISWS1617MuellerKimmeyer}) runterladen
\item Bei Eclipse unter File->Open Project from File System... und dann auf Directory... den Ordner ``fachhandlungClient'' in unserem Repository auswählen und auf ``Finish'' klicken
\item Unter Aquaapp/src/de.eis.muellerkimmeyer können nun die Java Dateien geöffnet werden
\item Nun müssen eventuell noch zwei Bibliotheken runtergeladen und in Eclipse eingebunden werden. Dafür muss man in Eclipse einen Rechtsklick auf das Projekt machen, dann auf Build Path -> Configure Build Path.... und dann unter Libraries auf Add JARs... klicken. Dort muss man dann die folgenden Dateien auswählen, die vorher runtergeladen werden müssen:
\begin{itemize}
\item java-json.jar: \url{http://www.java2s.com/Code/Jar/j/Downloadjavajsonjar.htm}
\item sqlite-jdbc-3.16.1.jar: \url{https://bitbucket.org/xerial/sqlite-jdbc/downloads}
\end{itemize}
\item Anwendung kann jetzt gestartet werden
\end{enumerate}

\subsubsection{Executable Jar File}

Im Implementations-Ordner befindet sich eine ausführbare Jar Datei der Kundenverwaltung sowie die Zugehörige Datenbank Datei. Die Jar Datei lässt sich ausführen und öffnet das Programm, allerdings werden die Bilder leider nicht angezeigt. Wenn die Anwendung über Eclipse gestartet wird, klappt es.

\section{Testdaten}

Es wurde bereits für die App ein Account angelegt. Man kann sich mit folgenden Login Daten einloggen: E-Mail: user1@web.de, Passwort: 123456. Auf diesem Account befinden sich schon ein paar Eintragungen für Wasserwerte. Dieser Account wurde auch schon bereits im Fachhandlung Client hinzugefügt.